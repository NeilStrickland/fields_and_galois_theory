%%Autogenerated from fields.tex: do not edit


\BeginDeferredSolution{ex-which-fields}{1.1}
 The set $K_0$ is not a field, because the element $1\in K_0$ has no
 additive inverse in $K_0$.  The set $K_1$ is a commutative ring but
 not a field, because the nonzero element $2\in K_1$ has no
 multiplicative inverse in $K_1$.

 The set $K_2$ (otherwise known as
 $\Q(\sqrt{2})$) is a field.  Indeed, it is clearly closed under
 addition and contains $0$ and $1$.  It is also closed under
 multiplication because for all $a,b,c,d\in\Q$ we have
 \[ (a+b\sqrt{2})(c+d\sqrt{2})=(ac+2bd)+(ad+bc)\sqrt{2} \]
 (and $ac+2bd,ad+bc\in\Q$).  Finally, any nonzero element
 $x\in\Q(\sqrt{2})$ has the form $x=a+b\sqrt{2}$ where at least one of
 $a$ and $b$ are nonzero.  A standard lemma tells us that $\sqrt{2}$
 is irrational, and thus that $a^2-2b^2$ cannot be zero.  It follows
 that the expression $y=(a-b\sqrt{2})/(a^2-2b^2)$ gives a well-defined
 element of $K_2$, and one checks directly that $xy=1$, so $y$ is a
 multiplicative inverse for $x$.  This proves that $K_2$ is a subfield
 of $\C$.

 Next, $K_3$ is just equal to $\R$, so it is a field.  The set $K_4$
 contains the element $\al=2^{1/3}$ but it does not contain $\al^2$,
 so it is not closed under multiplication, so it is not a field (or
 even a ring).  The set $K_4$ is a commutative ring, with the pair
 $(1,1)$ as the multiplicative identity.  However, it is not a field.
 Indeed, the element $e=(1,0)$ is nonzero but for any $(a,b)\in K_4$
 we have $e.(a,b)=(a,0)\neq(1,1)$; this shows that $e$ has no
 multiplicative inverse.  The set $K_6=\Z/6\Z$ is a commutative ring
 but not a field, because the nonzero element $\ov{2}$ has no
 inverse, as we see from the multiplication table modulo $6$:
 \[ \renewcommand{\arraystretch}{1.5}
   \begin{array}{|c||c|c|c|c|c|c|}
    \hline
     \cdot & 0 & 1 & 2 & 3 & 4 & 5 \\ \hline
     0     & 0 & 0 & 0 & 0 & 0 & 0 \\ \hline
     1     & 0 & 1 & 2 & 3 & 4 & 5 \\ \hline
     2     & 0 & 2 & 4 & 0 & 2 & 4 \\ \hline
     3     & 0 & 3 & 0 & 3 & 0 & 3 \\ \hline
     4     & 0 & 4 & 2 & 0 & 4 & 2 \\ \hline
     5     & 0 & 5 & 4 & 3 & 2 & 1 \\ \hline
   \end{array}
 \]
 On the other hand, the ring $\Z/7Z$ is a field.  Indeed, we have
 \[ 1^2 = 2\tm 4 = 3 \tm 5 = 6^2 = 1 \pmod{7}, \]
 so in $\Z/7\Z$ we have
 \[ 1^{-1} = 1 \qquad
    2^{-1} = 4 \qquad
    3^{-1} = 5 \qquad
    4^{-1} = 2 \qquad
    5^{-1} = 3 \qquad
    6^{-1} = 6,
 \]
 so every nonzero element has an inverse.  (The real reason for the
 difference between $K_6$ and $K_7$ is that $7$ is prime and $6$ is
 not.)
\EndDeferredSolution

\BeginDeferredSolution{ex-Ri-field}{1.2}
 We have $\F_2[i]=\{0,1,i,1+i\}$ and one can check directly that none
 of these elements is an inverse for $1+i$, so $\F_2[i]$ is not a
 field.  Alternatively $(1+i)^2=2i=0$ which would contradict
 Lemma~\ref{lem-domain} if $\F_2[i]$ were a field.

 Similarly, in $\F_5[i]$ we find that $2+i$ and $2-i$ are nonzero but
 $(2+i)(2-i)=5=0$, so again $\F_5[i]$ is not a field.

 Now consider $\F_3[i]$, and put $\al=1+i$.  We find that
 \begin{align*}
  \al^0 &= 1 & \al^1 &= 1+i \\
  \al^2 &= -i & \al^3 &= 1-i \\
  \al^4 &= -1 & \al^5 &= -1-i \\
  \al^6 &= i & \al^7 &= -1+i \\
  \al^8 &= 1.
 \end{align*}
 From this we see that every nonzero element of $\F_3[i]$ is $\al^k$
 for some $k\in\{0,\dotsc,7\}$, and that this has inverse
 $\al^{8-k}$.  This shows that $\F_3[i]$ is a field.
\EndDeferredSolution

\BeginDeferredSolution{ex-Qp-subfields}{1.3}
 Let $K$ be a subfield of $\Q(\sqrt{p})$.  This contains $1$ and is
 closed under addition and subtraction, so it must contain $\Z$.  For
 integers $b>0$ we then deduce that $b^{-1}\in K$, and so $a/b\in K$
 for all $a\in\Z$; this shows that $K$ contains $\Q$.  Suppose that
 $K$ is not equal to $\Q$; then $K$ must contain some element
 $\al=u+v\sqrt{p}$ with $u,v\in\Q$ and $v\neq 0$.  As $u\in\Q\sse K$
 and $\al\in K$ we see that the number $v\sqrt{p}=\al-u$ is also in
 $K$.  Similarly, we have $v^{-1}\in K$ and so
 $\sqrt{p}=v^{-1}.(v\sqrt{p})\in K$.  Finally, let $x$ and $y$ be
 arbitrary rational numbers; then $x,y,\sqrt{p}\in K$, so
 $x+y\sqrt{p}\in K$.  This proves that $K$ is all of $\Q(\sqrt{p})$,
 as required.
\EndDeferredSolution

\BeginDeferredSolution{ex-nth-root-aut}{1.4}
 Put $\al=a^{1/n}$, so the field in question is $K=\Q(\al)\sse\R$.
 Let $\sg\:K\to K$ be an automorphism, and put
 $\zt=\sg(\al)/\al\in K\sse\R$.  We can apply $\sg$ to the equation
 $\al^n=a$ to get $\sg(\al)^n=a$, and then divide by the original
 equation to get $\zt^n=1$.  As $\zt$ is real and $n$ is odd, we see
 that $\zt$ has the same sign as $\zt^n$, but $\zt^n=1>0$, so
 $\zt>0$.  We also have $(\zt-1)(1+\zt+\dotsb+\zt^{n-1})=\zt^n-1=0$,
 but all terms in the sum $1+\zt+\dotsb+\zt^{n-1}$ are strictly
 positive, so $\zt=1$.  This means that $\sg(\al)=\al$, so $\sg$ acts
 as the identity on $\Q(\al)=K$.
\EndDeferredSolution

\BeginDeferredSolution{ex-aut-F-four}{1.5}
 We have $\F_4=\{0,1,\al,\al^2\}$ with $\al^2=\al^{-1}=1+\al$.  Any
 automorphism $\phi\:\F_4\to\F_4$ must be a bijection and must satisfy
 $\phi(0)=0$ and $\phi(1)=1$, so either
 \begin{itemize}
  \item[(a)] $\phi(\al)=\al$ and $\phi(\al^2)=\al^2$; or
  \item[(b)] $\phi(\al)=\al^2$ and $\phi(\al^2)=\al$.
 \end{itemize}
 In case~(a) we see that $\phi$ is the identity.  All that is left
 is to check that case~(b) really does define an automorphism, or
 equivalently that $\phi(x+y)=\phi(x)+\phi(y)$ and
 $\phi(xy)=\phi(x)\phi(y)$ for all $x,y\in\F_4$.  One way to do this
 would be to just work through the sixteen possible pairs $(x,y)$.
 More efficiently, we can note that $\phi(x)=x^2$ for all $x\in\F_4$.
 (This is clear for $x=0$ or $x=1$ or $x=\al$; for the case $x=\al^2$
 we recall that $\al^3=1$ so
 $(\al^2)^2=\al^4=\al^3.\al=\al=\phi(\al^2)$.)  Given this, it is
 clear that $\phi(xy)=x^2y^2=\phi(x)\phi(y)$ for all $x$ and $y$.  We
 also have $\phi(x+y)=(x+y)^2=x^2+y^2+2xy=\phi(x)+\phi(y)+2xy$, but we
 are working in characteristic two so $2xy=0$ and so
 $\phi(x+y)=\phi(x)+\phi(y)$ as required.
\EndDeferredSolution

\BeginDeferredSolution{ex-equaliser}{1.6}
 This is very similar to Proposition~\ref{prop-fixed-subfield}.  We
 have $\phi(0_L)=0_M=\psi(0_L)$, so $0_L\in K$.  Similarly,  we
 have $\phi(1_L)=1_M=\psi(1_L)$, so $1_L\in K$.  If $a,b\in K$ then
 $\phi(a)=\psi(a)$ and $\phi(b)=\psi(b)$ so
 \begin{align*}
  \phi(a+b) &=\phi(a)+\phi(b)=\psi(a)+\psi(b)=\psi(a+b) \\
  \phi(a-b) &=\phi(a)-\phi(b)=\psi(a)-\psi(b)=\psi(a-b) \\
  \phi(ab)  &=\phi(a)\phi(b)=\psi(a)\psi(b)=\psi(ab),
 \end{align*}
 which shows that $a+b,a-b,ab\in K$.  Finally, if $a\in K^\tm$ then we
 can apply Proposition~\ref{prop-hom-inj}(a) to both $\phi$ and $\psi$
 to get
 \[ \phi(a^{-1}) = \phi(a)^{-1} = \psi(a)^{-1} =\psi(a^{-1}), \]
 which shows that $a^{-1}\in K$.  Thus, $K$ is a subfield as claimed.
\EndDeferredSolution

\BeginDeferredSolution{ex-product-ring}{1.7}
 Put $R=K_0\tm K_1$.  We recall that this is the set of all pairs
 $(a_0,a_1)$, where $a_0\in K_0$ and $a_1\in K_1$.  By hypothesis we
 are given an addition rule and a multiplication rule for elements of
 $K_0$, and an addition rule and a multiplication rule for elements of
 $K_1$.  We combine these in the obvious way to define addition and
 multiplication in $R$:
 \begin{align*}
  (a_0,a_1) + (b_0,b_1) &= (a_0+b_0,a_1+b_1) \\
  (a_0,a_1)(b_0,b_1) &= (a_0b_0,a_1b_1).
 \end{align*}
 The zero element of $R$ is the pair $(0,0)$, and the unit element is
 $(1,1)$.  Suppose we have three elements $a,b,c\in R$, say
 $a=(a_0,a_1)$ and $b=(b_0,b_1)$ and $c=(c_0,c_1)$.  By the
 associativity rule in $K_0$ we have $a_0+(b_0+c_0)=(a_0+b_0)+c_0$.
 By the associativity rule in $K_1$ we have
 $a_1+(b_1+c_1)=(a_1+b_1)+c_1$.  It follows that in $R$ we have
 \begin{align*}
  a+(b+c) &= (a_0,a_1) + ((b_0,b_1)+(c_0,c_1)) \\
   &= (a_0,a_1)+(b_0+c_0,b_1+c_1) \\
   &= (a_0+(b_0+c_0),a_1+(b_1+c_1)) \\
   &= ((a_0+b_0)+c_0,(a_1+b_1)+c_1) \\
   &= (a_0+b_0,a_1+b_1)+(c_0,c_1) \\
   &= ((a_0,a_1)+(b_0,b_1))+(c_0,c_1) = (a+b)+c.
 \end{align*}
 (The first, second, fourth and fifth steps here are just instances of
 the definition of addition in $R$; the third step uses the
 associativity rules in $K_0$ and $K_1$.)  Thus, addition in $R$ is
 associative.

 Similarly, the distributivity rule in $K_0$ tells us that
 $a_0(b_0+c_0)=a_0b_0+a_0c_0$.  The distributivity rule in $K_1$ tells
 us that $a_1(b_1+c_1)=a_1b_1+a_1c_1$.  It follows that in $R$ we have
 \begin{align*}
  a(b+c) &= (a_0,a_1)(b_0+c_0,b_1+c_1) \\
         &= (a_0(b_0+c_0),a_1(b_1+c_1)) \\
         &= (a_0b_0+a_0c_0,a_1b_1+a_1c_1) \\
         &= (a_0b_0,a_1b_1)+(a_0c_0,a_1c_1) = ab+ac.
 \end{align*}
 The other commutative ring axioms can be checked in the same way.

 As $1\neq 0$ in $K_0$, we see that the element $e=(1,0)\in R$ is
 nonzero.  For any element $a=(a_0,a_1)\in R$ we have
 $ea=(a_0,0)\neq(1,1)=1_R$, so $a$ is not inverse to $e$.  Thus $e$ is
 a nonzero element with no inverse, proving that $R$ is not a field.
\EndDeferredSolution

\BeginDeferredSolution{ex-which-linear}{2.1}
 \begin{itemize}
  \item $\phi_0$ is not linear because $\phi_0(-I)=(-I)^2=I\neq -\phi(I)$.
  \item $\phi_1$ is linear because
   \[ \phi_1(sA+tB)=sA+tB-(sA+tB)^T =
        sA+tB-sA^T-tB^T=s(A-A^T)+t(B-B^T)=s\phi_1(A)+t\phi_1(B).
   \]
   (This is enough by Remark~\ref{rem-linear}.)
  \item $\phi_2$ is also linear, because
   \[ \phi_2\left(s\bsm a\\ b\esm + t\bsm c\\ d\esm\right) =
       \phi_2\bsm sa+tc\\ sb+td\esm =
       (sa+tc)x+(sb+td)x^2 =
       s(ax+bx^2)+t(cx+dx^2) =
       s\phi_2\bsm a\\ b\esm + t\phi_2\bsm c\\ d\esm .
   \]
  \item $\phi_3$ is not linear, because
   \[ \phi_3\left(-\bsm 0\\ 1\esm\right) = \phi_3\bsm 0\\ -1\esm =
        (-x)^2 \neq -x^2 = -\phi_3\bsm 0\\ 1\esm.
   \]
  \item $\phi_4$ is linear, because if $h(x)=s\,f(x)+t\,g(x)$ then
   $h(2)=s\,f(2)+t\,g(2)$ and $h(-2)=s\,f(-2)+t\,g(-2)$ so
   \[ \phi_4(s\,f(x)+t\,g(x)) = \bsm h(2)\\ h(-2)\esm =
       s\bsm f(2)\\ f(-2)\esm + t\bsm g(2)\\ g(-2)\esm =
        s\phi_4(f(x)) + t\phi_4(g(x)).
   \]
  \item $\phi_5$ is not linear.  Indeed, for constant polynomials we
   just have $\phi_5(c)=c^3$, so
   $\phi_5(1+1)=8\neq 2=\phi_5(1)+\phi_5(1)$.
 \end{itemize}
\EndDeferredSolution

\BeginDeferredSolution{ex-degrees-possible}{2.2}
 No.  We would have
 \begin{align*}
  [M:\Q] &= [M:K][K:\Q] = 7\tm 3 = 21 \\
  [M:\Q] &= [M:L][L:\Q] = 5\tm 4 = 20,
 \end{align*}
 which is obviously not possible.
\EndDeferredSolution

\BeginDeferredSolution{ex-find-degrees}{2.3}
 Put $a=[L:K]$ and $b=[M:L]$ and $c=[N:M]$.  As $K$, $L$, $M$ and $N$
 are all different we must have $a,b,c>1$.  We also have
 \begin{align*}
  ab &= [M:L][L:K] = [M:K] = 6 \\
  bc &= [N:M][M:L] = [N:L] = 15.
 \end{align*}
 As $ab=6$ with $a,b>1$ we must have $(a,b)=(2,3)$ or $(a,b)=(3,2)$.
 As $bc=15$ with $b,c>1$ we must have $(b,c)=(3,5)$ or $(b,c)=(5,3)$.
 The only way these can both be satisfied is if $(a,b,c)=(2,3,5)$.
\EndDeferredSolution

\BeginDeferredSolution{ex-basis-i}{2.4}
 The general form for elements of $V$ is
 \[ M = \bsm a & b & c \\ b & d & e \\ c & e & -a-d \esm =
     aA+bB+cC+dD+eE,
 \]
 where
 \[ A = \bsm 1&0&0\\ 0&0&0\\ 0&0&-1 \esm
    B = \bsm 0&1&0\\ 1&0&0\\ 0&0&0 \esm
    C = \bsm 0&0&1\\ 0&0&0\\ 1&0&0 \esm
    D = \bsm 0&0&0\\ 0&1&0\\ 0&0&-1 \esm
    E = \bsm 0&0&0\\ 0&0&1\\ 0&1&0 \esm.
 \]
 It follows easily from this that the list $A,B,C,D,E$ is a basis for
 $V$.
\EndDeferredSolution

\BeginDeferredSolution{ex-matrix-subspaces}{2.5}
 \begin{itemize}
  \item Put $A=iI=\bsm i&0\\ 0&i\esm$.  As $\ov{i}=-i$ we see that
   $A^\dag=-A$, so $A\in V$.  On the other hand, we have $-iA=I$ and
   $I+I^\dag=2I$ so $-iA\not\in V$.  This means that $V$ is not closed
   under multiplication by the complex number $-i$, so it is not a
   subspace over $\C$ of $M_2(\C)$.
  \item If $A=\bsm a&b\\ c&d\esm$ then
   $A+A^\dag=\bsm a+\ov{a} & b+\ov{c}\\ c+\ov{b} & d\ov{d}\esm$.  For
   this to be zero, we need $a+\ov{a}=d+\ov{d}=0$ (so $a$ and $d$ are
   purely imaginary) and $c=-\ov{b}$.  Equivalently, $A$ must have the
   form
   \[ A=\bsm iw & x+iy \\ -x+iy & iz \esm =
       w\bsm i&0\\0&0 \esm +
       x\bsm 0&1\\-1&0\esm +
       y\bsm 0&i\\ i&0\esm +
       z\bsm 0&0\\ 0&i\esm
   \]
   for some $w,x,y,z\in\R$.  It follows that $V$ is a subspace over
   $\R$ of $M_2(\C)$, with basis given by the matrices
   \[  \bsm i&0\\0&0 \esm \hspace{3em}
       \bsm 0&1\\-1&0\esm \hspace{3em}
       \bsm 0&i\\ i&0\esm \hspace{3em}
       \bsm 0&0\\ 0&i\esm.
   \]
   In particular, this basis has size four, so $\dim_\R(V)=4$ as
   required.
 \end{itemize}
\EndDeferredSolution

\BeginDeferredSolution{ex-rational-extension}{2.6}
 As $L$ is generated over $\C$ by $x$, it is certainly generated over
 the larger field $K$ by $x$.  Put $f(t)=t^n-x^n\in K[t]$.  Clearly
 $f(x)=0$, so $x$ is algebraic over $K$.  Let $g(t)$ be the minimal
 polynomial of $x$ over $K$, so $g(t)$ divides $f(t)$, and
 $L=K(x)\simeq K[t]/g(t)$, so $m=[L:K]$ is the degree of $g(t)$.  As
 $g(t)$ divides $f(t)$ we see that $m\leq n$.  We will suppose that
 $m<n$ and derive a contradiction; this will complete the proof.

 The coefficients of $g(t)$ are elements of $K=\Q(x^n)$, so they can
 be written as $a_i(x^n)/b_i(x^n)$ for certain polyomials $a_i(s)$ and
 $b_i(s)\neq 0$.  If we let $d(s)$ be the product of all the terms
 $b_i(s)$ we obtain an expression
 $d(x^n)g(t)=\sum_{i=0}^mc_i(x^n)t^i$, with $c_i(s),d(s)\in\C[s]$.  By
 assumption $g(x)=0$, so $\sum_{i=0}^mc_i(x^n)x^i=0$.  As $m<n$ we can
 compare coefficient of $x^{nj+i}$ (for $0\leq i\leq m$) to see that
 $c_i(x)=0$.  It follows that $g(t)=0$, which contradicts the fact
 that $g(t)$ divides $f(t)$, as required.
\EndDeferredSolution

\BeginDeferredSolution{ex-F-four-ideal}{3.1}
 Recall that $\F_4=\{0,1,\al,\al^2\}$ with $\al^2=\al^{-1}=1+\al$.
 Define $\phi\:\Z[x]\to\F_4$ by
 \[ \phi(a_0+a_1x+\dotsb+a_dx^d) =
     \ov{a_0}+\ov{a_1}\al+\dotsb+\ov{a_d}\al^d.
 \]
 This is clearly a homomorphism.  It satisfies $\phi(0)=0$ and
 $\phi(1)=1$ and $\phi(x)=\al$ and $\phi(x^2)=\al^2$, so every element
 of $\F_4$ is in the image of $\phi$, so $\phi$ is surjective.  Let
 $I$ be the kernel of $\phi$.  Proposition~\ref{prop-induced-hom} then
 gives us an induced isomorphism $\ov{\phi}\:\Z[x]/I\to\F_4$.  One can
 check that $I$ can be described more explicitly as
 \[ I = \{f(x)\in\Z[x]\st f(x)=2g(x)+(x^2+x+1)h(x)
           \text{ for some } g(x),h(x)\in\Z[x]\}.
 \]
\EndDeferredSolution

\BeginDeferredSolution{ex-ideals-twelve}{3.2}
 Write
 \[ R=\Z/12\Z=\{0,1,2,3,4,5,6,7,8,9,10,11\}. \]
 The principal ideals are as follows:
 \begin{align*}
  R.0 &= \{0\} \\
  R.1 &= \{0,1,2,3,4,5,6,7,8,9,10,11\} = R.5 = R.7 = R.11 \\
  R.2 &= \{0,2,4,6,8,10\} = R.10 \\
  R.3 &= \{0,3,6,9\} = R.9 \\
  R.4 &= \{0,4,8\} = R.8 \\
  R.6 &= \{0,6\}.
 \end{align*}
 In fact, it can be shown that every ideal in $\Z/n\Z$ is principal,
 so the above list actually contains all ideals in $R$.
\EndDeferredSolution

\BeginDeferredSolution{which-irreducible}{4.1}
 We first recall Eisenstein's criterion. Suppose we have a monic
 polynomial
 \[ a_0 + a_1x + \dotsb + a_{d-1}x^{d-1} + x^d \]
 and a prime number $p$ such that
 \begin{itemize}
  \item[(a)] the coefficients $a_0,\dotsc,a_{d-1}$ are all integers
   divisible by $p$; and
  \item[(b)] the constant term $a_0$ is not divisible by $p^2$,
 \end{itemize}
 then $g(x)$ is irreducible over $\Q$.  We find the $f_0(x)$ is
 irreducible by Eisenstein's criterion with $p=3$, and that $f_3(x)$
 is irreducible by Eisenstein's criterion with $p=5$.  On the other
 hand, $f_1(x)=(x-2)(x^2+x+1)$ and $f_2(x)=(x-3)(x+6)$, so neither of
 these is irreducible over $\Q$.
\EndDeferredSolution

\BeginDeferredSolution{ex-euclid}{4.2}
 We start with $f_0(x)=f(x)$ and
 $f_1(x)=f'(x)/4=x^3+\tfrac{3}{2}x^2+\tfrac{3}{2}x+\tfrac{1}{2}$.  By
 long division we have
 \[ f_0(x) = (x+\tfrac{1}{2})f_1(x) +
      (\tfrac{3}{4}x^2+\tfrac{3}{4}x+\tfrac{3}{4}),
 \]
 so $f_2(x)=x^2+x+1$.  We then divide $f_1(x)$ by $f_2(x)$ and obtain
 \[ f_1(x) = (x+\tfrac{1}{2}) f_2(x) \]
 (with no remainder).  Thus the algorithm stops with
 $\gcd(f(x),f'(x))=x^2+x+1$.  This means that every root of $x^2+x+1$
 is a double root of $f(x)$, so $f(x)$ is divisible by $(x^2+x+1)^2$,
 but these are monic polynomials of the same degree, so
 $f(x)=(x^2+x+1)^2$.
\EndDeferredSolution

\BeginDeferredSolution{ex-eisenstein-shift}{4.3}
 The polynomial $f(x+2)=x^4+3x^3+3x^2+3x+3$ satisfies Eisenstein's
 criterion at $p=3$, so $f(x+2)$ is irreducible, so $f(x)$ is
 irreducible.  We can also make the same argument using
 $f(x-1)=x^4-9x^3+30x^2-42x+21$ (but $f(x+1)$ does not work).
\EndDeferredSolution

\BeginDeferredSolution{ex-modular-irreducible}{4.4}
 First, in $\F_2$ we have $f(0)=1$ and $f(1)=1$, so $f(x)$ has no
 roots, so it has no factors of degree one.  Thus, the only way it
 could factorise would be as an irreducible quadratic times an
 irreducible cubic.  The only quadratics over $\F_2$ are $x^2$,
 $x^2+1=(x+1)^2$, $x^2+x=x(x+1)$ and $x^2+x+1$.  Only the last of
 these is irreducible.  We find by long division over $\F_2$ that
 \[ f(x) = (x^3+x^2)(x^2+x+1) + 1, \]
 so $f(x)$ is not divisible by $x^2+x+1$.  It is therefore irreducible
 as claimed.

 Now suppose we have a factorisation $f(x)=g(x)h(x)$ in $\Q[x]$, where
 $g(x)$ and $h(x)$ are monic.  We see from Gauss's Lemma that
 $g(x),h(x)\in\Z[x]$, so it makes sense to reduce everything modulo
 $2$.  We then have $\ov{f}(x)=\ov{g}(x)\ov{h}(x)$ in $\F_2[x]$,
 but $\ov{f}(x)$ is irreducible, so one of the factors must be equal
 to one, say $\ov{g}(x)=1$.  As $g(x)$ is monic, the only way we can
 have $\ov{g}(x)=1$ is if $g(x)=1$.  We deduce that $f(x)$ is
 irreducible in $\Q[x]$, as claimed.
\EndDeferredSolution

\BeginDeferredSolution{ex-x-to-the-p}{4.5}
 I claim that $R$ is just the ring $\F_p[x^p]$ of polynomials in
 $x^p$.  To see this, consider an arbitrary element $f(x)\in\F_p[x]$,
 say $f(x)=\sum_{i=0}^Na_ix^i$ for some sequence of coefficients
 $a_i\in\F_p$.  We then have $f'(x)=\sum_{i=0}^N i\,a_i\,x^{i-1}$, so
 $f'(x)=0$ iff $i\,a_i=0$ for all $i$.  If $i$ is divisible by $p$
 then it gives the zero element of $\F_p$ so the equation $i\,a_i=0$
 holds automatically.  However, if $i$ is not divisible by $p$
 then it gives a nonzero element of the field $\F_p$, so we can
 multiply by the inverse to get $a_i=0$.  It follows that $f'(x)=0$
 iff $f(x)$ has the form $\sum_{j=0}^Ma_{jp}x^{jp}$ say, or
 equivalently $f(x)$ is a polynomial function of $x^p$.
\EndDeferredSolution

\BeginDeferredSolution{ex-splitting-misc-i}{5.1}
 We will write $K_i$ for the splitting field of $f_i(x)$.
 \begin{itemize}
  \item We can write $f_0(x)$ as $(x-1)^2$, so $K_0=\Q$.
  \item We can factor $f_1(x)$ as $(x^2-2)(x^2-3)$, so the roots are
   $\pm\sqrt{2}$ and $\pm\sqrt{3}$, so the
   $K_1=\Q(\sqrt{2},\sqrt{3})$.
  \item The roots of $f_2(x)$ are $(1\pm\sqrt{-3})/2$, so
   $K_2=\Q(\sqrt{-3})$.
  \item The roots of $f_3(x)$ are $\al$, $\om\al$ and $\om^2\al$,
   where $\al$ is the real cube root of $2$, and
   $\om=e^{2\pi i/3}=(\sqrt{-3}-1)/2$.  It follows that $K_3$ contains
   $\al$ and $\om\al$, so it also contains $(\om\al)/\al=\om$, so it
   also contains $2\om+1=\sqrt{-3}$.  Form this it follows that
   $K_3=\Q(\al,\om)=\Q(\al,\sqrt{-3})$.
  \item We can regard $f_4(x)$ as a quadratic function of $x^2$, and
   we find that it vanishes when $x^2=(4\pm\sqrt{12})/2=2\pm\sqrt{3}$,
   so $x=\pm\sqrt{2\pm\sqrt{3}}$.  Thus, one root of $f(x)$ is
   $\al=\sqrt{2+\sqrt{3}}$, and another is $-\al$.  The other two
   roots are $\bt$ and $-\bt$, where $\bt=\sqrt{2-\sqrt{3}}$.
   However, we have
   $\al\bt=\sqrt{(2+\sqrt{3})(2-\sqrt{3})}=\sqrt{1}=1$, so
   $\bt=\al^{-1}$.  It follows that the full list of roots is
   $\al,-\al,1/\al,-1/\al$, so $K_4=\Q(\al)$.
  \item If we let $\al$ denote the positive real fourth root of $2$,
   then the roots of $f_5(x)$ are $\al,i\al,-\al$ and $-i\al$.  It
   follows that $K_5=\Q(\al,i)$.  It follows that $[K_5:\Q]=8$.
  \item The roots of $f_6(x)$ are the $6$th roots of unity, which are
   the powers of $\al=e^{\pi i/3}=(1+\sqrt{-3})/2$, so
   $K_6=\Q(\sqrt{-3})$.
  \item The roots of $f_7(x)$ are the numbers $2\al^k$, where again
   $\al=e^{\pi i/3}=(1+\sqrt{-3})/2$.  It follows that
   $K_7=K_6=\Q(\sqrt{-3})$.
 \end{itemize}
\EndDeferredSolution

\BeginDeferredSolution{ex-splitting-misc-ii}{5.2}
\ \\
 \begin{itemize}
  \item[(a)] The roots of $x^4+1$ are the primitive 8th roots of unity.
   One of these is $\al=e^{i\pi/4}=(1+i)/\sqrt{2}$, and the others are
   $\al^3=i\al$, $\al^5=-\al$ and $\al^7=-i\al$.  Note that $i=\al^2$
   and $\sqrt{2}=(1+i)/\al=(1+\al^2)/\al$, so $i,\sqrt{2}\in\Q(\al)$.
   It is also clear that $\al\in\Q(i,\sqrt{2})$, so the relevant
   splitting field is $\Q(i,\sqrt{2})$.

  \item[(b)] We may observe that $x^4+x^2+1=(x^2+x+1)(x^2-x+1)$, and
   so its roots are just the roots of the two quadratic factors. These
   are
   \[ \frac{-1\pm\sqrt{-3}}{2} \qquad\mbox{and}\qquad
      \frac{ 1\pm\sqrt{-3}}{2}.
   \]
   It follows that the splitting field is $\Q(\sqrt{-3})$, of degree 2
   over $\Q$.

  \item[(c)] The roots of $x^6+1$ are the 6th roots of $-1$.  As
   $-1=e^{i\pi}$, one of these roots is
   \[ \al=e^{i\pi/6}=(\sqrt{3}+i)/2. \]
   The other roots are obtained by multiplying $\al$ by a 6th root of
   $1$, but the 6th roots of $1$ are just the powers of $\al^2$, so
   the roots of $x^6+1$ are $\al,\al^3,\al^5,\al^7,\al^9$ and
   $\al^{11}$.  Thus, the splitting field is just $\Q(\al)$.  Note
   that $\al\in\Q(i,\sqrt{3})$, but $i=e^{i\pi/2}=\al^3\in\Q(\al)$,
   and so $\sqrt{3}=2\al-i\in\Q(\al)$.  It follows that the splitting
   field can also be described as $\Q(i,\sqrt{3})$.  It therefore has
   degree $4$ over $\Q$.

  \item[(d)] Note that $x^9-1=(x^3-1)(x^6+x^3+1)$, so the roots of
   $x^6+x^3+1$ are the primitive 9th roots of unity. One may then
   observe that if $\zt$ is a primitive 9th root of unity, all other
   primitive 9th roots of unity are powers of $\zt$, so that the
   splitting field is just $\Q(\zt)$. Its degree over $\Q$ is just the
   degree of the minimal polynomial of $\zt$, but this is the given
   polynomial $x^6+x^3+1$ as it is irreducible (substitute
   $x\mapsto x+1$ and use Eisenstein with $p=3$). So $[\Q(\zt):\Q]=6$.

   Alternatively, the roots of $y^2+y+1$ are
   $\om=\frac{-1+\sqrt{-3}}{2}\in\Q(\sqrt{-3})$ and
   $\om^{-1}=\om^2=\frac{-1-\sqrt{-3}}{2}\in\Q(\sqrt{-3})$.  The roots
   of $x^6+x^3+1$ are the cube roots of these, so if
   $\al=\om^{\frac{1}{3}}$, then the roots are
   $\al,\om\al,\om^2\al;\al^{-1},\om\al^{-1},\om^2\al^{-1}$. So the
   splitting field is $\Q(\al,\om)$; but $\Q(\om)=\Q(\sqrt{-3})$, so
   has degree 2 over $\Q$. Further, $\al$ satisfies the cubic equation
   $x^3-\om$ with coefficients in $\Q(\om)$, so $\Q(\om,\al)$ has
   degree at most 3 over $\Q(\om)$. Thus the degree of the splitting
   field is at most 6 over $\Q$ (using the Degrees Theorem). On the
   other hand, the polynomial is irreducible (as above), so adjoining
   any root of it gives a field extension of degree 6, and so
   adjoining all the roots gives a field extension of degree at least
   6. Thus the degree equals 6.
 \end{itemize}
\EndDeferredSolution

\BeginDeferredSolution{ex-transcendental}{5.3}
 First define $\chi_0\:K[x]\to L$ by $\chi_0(p(x))=p(\al)$, or more
 explicitly
 \[ \chi_0(\sum_ia_ix^i) = \sum_ia_i\al^i. \]
 The kernel of this is $I(\al,K)$, which is zero because $\al$ is
 transcendental.  Thus, if $q(x)\neq 0$ we see that $q(\al)$ is a
 nonzero element of $L$, so it has an inverse in $L$.  Thus, given a
 rational function $f(x)=p(x)/q(x)$, we can try to define
 $\chi(f(x))=p(\al)/q(\al)\in L$.  There is a potential ambiguity
 here: what if $f(x)$ can be represented in a different way, say as
 $f(x)=r(x)/s(x)$ for some $r(x),s(x)\in K[x]$ with $s(x)\neq 0$?  By
 the construction of $K(x)$, this means that $p(x)s(x)=r(x)q(x)$ in
 $K[x]$, which implies that $p(\al)s(\al)=r(\al)q(\al)$ in $L$, which
 means that $p(\al)/q(\al)=r(\al)/s(\al)$ in $L$.  We therefore have a
 well-defined function $\chi\:K(x)\to L$ as described.  We know from
 Proposition~\ref{prop-hom-inj} that $\chi(K(x))$ is a subfield of $L$
 and that $\chi$ gives an isomorphism $K(x)\to\chi(K(x))$, so it will
 suffice to show that $\chi(K(x))=K(\al)$.  It is clear that
 $K=\chi(K)\sse\chi(K(x))$  and $\al=\chi(x)\in\chi(K(x))$, and by
 definition $K(\al)$ is the smallest subfield of $L$ containing $K$
 and $\al$, so $K(\al)\sse\chi(K(x))$.  Conversely, as $K(\al)$ is a
 field containing $K$ and $\al$, we see that it must contain all
 powers of $\al$, and then all $K$-linear combinations of powers;
 equivalently, it must contain $q(\al)$ for all $q\in K[x]$.  If
 $q(x)$ is nonzero then $q(\al)\in K(\al)\sm\{0\}=K(\al)^\tm$, so
 $1/q(\al)\in K(\al)$, so $p(\al)/q(\al)\in K(\al)$ for all
 $p(x)\in K[x]$.  This shows that $K(\al)$ contains $\chi(K(x))$, so
 we must have $K(\al)=\chi(K(x))$, as required.
\EndDeferredSolution

\BeginDeferredSolution{ex-cayley}{5.4}
 We can define a function $\mu\:L\to L$ by $\mu(a)=\al a$ for all
 $a\in L$.  This is clearly $K$-linear (or even $L$-linear, but we
 will not use that).  Let $f(t)\in K[t]$ be the characteristic
 polynomial of $\mu$.  More explicitly, we can choose a basis
 $e_1,\dotsc,e_d$ for $L$ over $K$, and note that there must be
 elements $A_{ij}\in K$ with $\mu(e_i)=\al e_i=\sum_jA_{ij}e_j$ for
 all $i$.  This gives a matrix $A\in M_d(K)$, and thus a matrix
 $tI-A\in M_d(K[t])$.  We then have $f(t)=\det(tI-A)$, which is a
 monic polynomial of degree $d$ over $K$, so it can be written as
 $\sum_{i=0}^dc_it^i$ for some coefficients $c_i\in K$.  The
 Cayley-Hamilton theorem then tells us that
 $\sum_{i=0}^dc_i\mu^i=f(\mu)=0$ as a $K$-linear map from $L$ to $L$.
 As $\mu(a)=\al a$ (and so $\mu^2(a)=\mu(\al a)=\al^2 a$, and so on)
 we deduce that $\sum_{i=0}^dc_i\al^ia=\sum_{i=0}^dc_i\mu^i(a)=0$.  In
 particular, we can take $a=1$ and thus deduce that $f(\al)=0$, so
 $f(x)\in I(\al,K)$.  As $f$ is monic we also have $f(x)\neq 0$, so
 $I(\al,K)\neq 0$ as claimed.
\EndDeferredSolution

\BeginDeferredSolution{ex-Q-bar}{5.5}
 \begin{itemize}
  \item[(a)] If $\al\in\ov{\Q}$ then
   Proposition~\ref{prop-simple-algebraic} tells us that
   $[\Q(\al):\Q]=\deg(\min(\al,\Q))<\infty$.  If $[\Q(\al):\Q]<\infty$
   then evidently $\Q(\al)$ is an example of a subfield $K\sse\C$ with
   $\al\in K$ and $[K:\Q]<\infty$.  If we are given such a field $K$,
   then Proposition~\ref{prop-finite-algebraic} (applied to the
   extension $\Q\subset K$) tells us that $\al\in\ov{\Q}$.  Thus, the
   three conditions mentioned are all equivalent.
  \item[(b)] First, it is clear that $\ov{\Q}$ contains $\Q$, so
   $0,1\in\ov{\Q}$.  Suppose that $\al,\bt\in\ov{\Q}$.  This means
   that there are subfields $L,M\subset\C$ with $\al\in L$ and
   $\bt\in M$ and $[L:\Q],[M:\Q]<\infty$.  Now
   Proposition~\ref{prop-subfield-join} tells us that $LM$ is a
   subfield of $\C$ containing both $\al$ and $\bt$, such that
   $[LM:\Q]<\infty$.  As (iii) implies (i) above, we see that
   $LM\sse\ov{\Q}$.  Now $\al+\bt$, $\al-\bt$ and $\al\bt$ all lie in
   $LM$, so they lie in $\ov{\Q}$.  Similarly, if $\al\neq 0$ then
   $\al^{-1}\in L\sse LM\sse\ov{\Q}$.  It follows that $\ov{\Q}$ is a
   subfield as claimed.
  \item[(c)] Now suppose that $\al\in\C$ and $\al$ is algebraic over
   $\ov{\Q}$.  We thus have a minimal polynomial
   $f(x)=\min(\al,\ov{\Q})(x)=\sum_{i=0}^da_ix^i$, with $a_d=1$ and
   $a_i\in\ov{\Q}$ for all $i$.  Now part~(a) tells us that there
   exists a field $L_i\subset\C$ with $a_i\in L_i\subset\C$ and
   $[L_i:\Q]<\infty$.  Put $L=L_0L_1\dotsb L_d$, so
   Proposition~\ref{prop-subfield-join} tells us that
   $[L:\Q]<\infty$.  Moreover, as $f(\al)=0$ we see that
   $[L(\al):L]\leq d$, so $[L(\al):\Q]=[L(\al):L][L:\Q]<\infty$.  This
   means that $L(\al)$ is a finite degree extension of $\Q$ containing
   $\al$, so $\al\in\ov{\Q}$ by criterion~(iii) above.
  \item[(d)] Suppose we have a nonconstant polynomial
   $f(x)\in\ov{\Q}[x]$.  We can regard this as a nonconstant polynomial
   over $\C$, so the Fundamental Theorem of Algebra tells us that
   there is a root (say $\al$) in $\C$.  Now the relation $f(\al)=0$
   tells us that $\al$ is algebraic over $\ov{\Q}$, so part~(c) tells
   us that $\al\in\ov{\Q}$.  We therefore see that any nonconstant
   polynomial over $\ov{\Q}$ has a root in $\ov{\Q}$, which means that
   $\ov{\Q}$ is algebraically closed.
 \end{itemize}
\EndDeferredSolution

\BeginDeferredSolution{ex-F-sixteen}{5.6}
 \begin{itemize}
  \item[(a)] It is a general fact that if $\tht$ is algebraic over $K$
   and the minimal polynomial has degree $d$, then the set
   $\{1,\tht,\dotsc,\tht^{d-1}\}$ is a basis for $K(\tht)$ over $K$.
   From this it follows that $\{1,\al\}$ is a basis for $\F_4$ over
   $\F_2$.  This means that every element of $\F_4$ can be written as
   $a_0+a_1\al$ for some $a_0,a_1\in\F_2=\{0,1\}$, so
   \[ \F_4 = \{0,1,\al,1+\al\}. \]
  \item[(b)] Similarly, as the minimal polynomial of $\bt$ over $\F_2$
   has degree $4$ we see that the set $\{1,\bt,\bt^2,\bt^3\}$ is a
   basis for $\F_{16}$ over $\F_2$.  This gives the following list of
   elements of $\F_{16}$:
   \begin{align*}
     & 0,1,\bt,1+\bt,\bt^2,1+\bt^2,\bt+\bt^2,1+\bt+\bt^2, \\
     & \bt^3,1+\bt^3,\bt+\bt^3,1+\bt+\bt^3,\bt^2+\bt^3,
       1+\bt^2+\bt^3,\bt+\bt^2+\bt^3,1+\bt+\bt^2+\bt^3.
   \end{align*}
  \item[(c)] As the minimal polynomial of $\bt$ is $t^4+t^3+t^2+t+1$,
   we have $\bt^4+\bt^3+\bt^2+\bt+1=0$.  If we multiply by $\bt-1$ and
   cancel we get $\bt^5-1=0$, so $\bt^5=1$.
  \item[(d)] The homomorphisms from $\F_4$ to $\F_{16}$ biject with
   the roots of the minimal polynomial $g(t)=t^2+t+1=0$ in $\F_{16}$.
   As this polynomial has degree two, it can have at most two roots in
   any field.  Thus, if we can find two roots then we need not look
   for any more.  By working through our list of elements of $\F_{16}$
   we find that the required roots are as follows:
   \begin{align*}
    \gm &= \bt^2+\bt^3 \\
    \dl &= 1+\bt^2+\bt^3.
   \end{align*}
   Indeed, we have
   \[ g(\gm) = 1 + \gm + \gm^2
       = 1 + \bt^2 + \bt^3 + (\bt^4 + 2\bt^2\bt^3 + \bt^6).
   \]
   We can discard the term $2\bt^2\bt^3$ because $2=0$ in $\F_2$.  We
   also know that $\bt^5=1$, so $\bt^6=\bt$.  Using these the above
   equation simplifies to $g(\gm)=1+\bt+\bt^2+\bt^3+\bt^4$, but this
   is just the minimal polynomial evaluated at $\bt$, so $g(\gm)=0$.
   A similar argument shows that $g(\dl)=0$ as well.  It follows that
   the two homomorphisms $\phi,\psi\:\F_4\to\F_{16}$ are given by
   \begin{align*}
    \phi(a_0+a_1\al) &= a_0 + a_1\gm = a_0 + a_1\bt^2 + a_1\bt^3 \\
    \psi(a_0+a_1\al) &= a_0 + a_1\dl = a_0 + a_1 + a_1\bt^2 + a_1\bt^3.
   \end{align*}
 \end{itemize}
\EndDeferredSolution

\BeginDeferredSolution{ex-abelian-transitive}{6.1}
 Suppose that $\sg(i)=i$.  Transitivity means that for any $j\in N$ we
 can choose $\tau\in A$ with $\tau(i)=j$.  As $A$ is commutative we
 then have
 \[ \sg(j) = \sg(\tau(i)) = \tau(\sg(i)) = \tau(i) = j. \]
 As $j$ was arbitrary, this means that $\sg$ is the identity.
 Thus the action is free, as claimed.

 Next, as $A$ is transitive we can choose $\sg_i\in A$ (for
 $i=1,\dotsc,N$) such that $\sg_i(1)=i$.  Now let $\tau$ be any
 element of $A$.  Put $i=\tau(1)$, and note that $\tau^{-1}\sg_i$
 sends $1$ to $1$.  As the action is free this means that
 $\tau^{-1}\sg_i=1$, so $\tau=\sg_i$.  This means that
 $A=\{\sg_1,\dotsc,\sg_n\}$, and these elements are all different,
 so $|A|=n$.
\EndDeferredSolution

\BeginDeferredSolution{ex-root-sqrt}{6.2}
\ \\
 \begin{itemize}
  \item[(a)] Suppose that $f(x^2)$ is irreducible.  If $f(x)=u(x)v(x)$
   then $f(x^2)=u(x^2)v(x^2)$, and as $f(x^2)$ is irreducible this
   means that either $u(x^2)$ or $v(x^2)$ is constant, so either
   $u(x)$ or $v(x)$ is constant.  This proves that $f(x)$ is
   irreducible.
  \item[(b)] The polynomial $f(x)=\vph_3(x)=x^2+x+1$ is irreducible,
   but one can check directly that $f(x^2)=f(x)f(-x)$, which shows
   that $f(x^2)$ is reducible.
  \item[(c)] Let $\al_1,\dotsc,\al_d$ be the roots of $f(x)$ in $\C$.
   As $f(x)$ has degree greater than one and is irreducible, it cannot
   be divisible by $x$, so we must have $\al_i\neq 0$ for all $i$.
   Choose a square root $\bt_i$ for $\al_i$.  We then have
   $f(x)=\prod_i(x-\al_i)=\prod_i(x-\bt_i^2)$, so
   \[ f(x^2) = \prod_i(x^2-\bt_i)^2 = \prod_i(x-\bt_i)(x-(-\bt_i)). \]
   It follows that $L=\Q(\bt_1,\dotsc,\bt_d)$ and
   \[ K = \Q(\al_1,\dotsc,\al_d) =
       \Q(\bt_1^2,\dotsc,\bt_d^2) \sse L.
   \]
   As both $K$ and $L$ are normal over $\Q$, we know that $G(L/K)$ is
   a normal subgroup of $G(L/\Q)$, and that
   $G(L/\Q)/G(L/K)\simeq G(K/\Q)$.  For $\sg\in G(L/K)$ we know that
   $\sg(\bt_i)^2=\al_i=\bt_i^2$, so $\sg(\bt_i)/\bt_i\in\{1,-1\}$.  We
   define $\chi_i(\sg)=\sg(\bt_i)/\bt_i$; it is not hard to check that
   this gives a group homomorphism $\chi_i\:G(L/K)\to\{1,-1\}$.  We
   can put these together to define a map $\chi\:G(L/K)\to\{1,-1\}^d$
   by $\chi(\sg)=(\chi_1(\sg),\dotsc,\chi_d(\sg))$.  As the elements
   $\bt_i$ generate $L$ over $K$, we see that $\chi$ is injective,
   so $G(L/K)$ is an elementary abelian $2$-group.  We cannot say much
   more than this without more information about the polynomial
   $f(x)$.
 \end{itemize}
\EndDeferredSolution

\BeginDeferredSolution{ex-sqrt-chain}{6.3}
 Put $\al_1=\sqrt{1111}$ and $\al_2=\sqrt{11+\al_1}$ and
 $\al_3=\sqrt{111+\al_2}$, so $K_i=K_{i-1}(\al_i)$.
 \begin{itemize}
  \item[(a)] Homomorphisms $\phi_1\:K_1\to\R$ biject with roots in
   $\R$ of the polynomial
   $\min(\al_1,K_0)(t)=\min(\sqrt{1111},\Q)(t)=t^2-1111$.  These roots
   are $\al_1\simeq 33.332$ and $-\al_1\simeq -33.332$.  More
   explicitly, there are two possible homomorphisms, namely
   \begin{align*}
    \phi_{11}(u+v\al_1) &= u+v\al_1 \\
    \phi_{12}(u+v\al_1) &= u-v\al_1.
   \end{align*}
  \item[(b)] The minimal polynomial of $\al_2$ over $K_1$ is
   $t^2-11-\al_1$.  If we apply $\phi_{11}$ to the coefficients of this,
   we just get the polynomial $t^2-11-\al_1$ again.  The extensions of
   $\phi_{11}$ biject with the roots in $\R$ of this polynomial, which
   are $\al_2\simeq 6.658$ and $-\al_2\simeq -6.658$.  More
   explicitly, there are two possible extensions of $\phi_{11}$,
   given by
   \begin{align*}
    \phi_{21}(u+v\al_2) &= u+v\al_2 \\
    \phi_{22}(u+v\al_2) &= u-v\al_2
   \end{align*}
   for all $u,v\in K_1$.  Alternatively, we can look back at the proof
   of the degree formula $[K_2:K_0]=[K_2:K_1][K_1:K_0]=2\tm 2=4$ and
   see that the list $1,\al_1,\al_2,\al_1\al_2$ is a basis for $K_2$
   over $\Q$.  In terms of this basis, we have
   \begin{align*}
    \phi_{21}(u+v\al_1+w\al_2+x\al_1\al_2) &= u+v\al_1+w\al_2+x\al_1\al_2 \\
    \phi_{22}(u+v\al_1+w\al_2+x\al_1\al_2) &= u+v\al_1-w\al_2-x\al_1\al_2
   \end{align*}
   for all $u,v,w,x\in\Q$.  Now consider instead extensions of the
   homomorphism $\phi_{12}$.  These again biject with the roots in
   $\R$ of a certain polynomial.  To find the required polynomial, we
   take $\min(\al_2,K_1)(t)=t^2-11-\al_1$ and apply $\phi_{12}$ to the
   coefficients, giving $t^2-11+\al_1$.  Here $11-\al_1\simeq
   -22.332<0$, so there are no such roots.  This means that
   the homomorphism $\phi_{12}\:K_1\to\R$ cannot be extended over
   $K_2$.
  \item[(c)] The minimal polynomial of $\al_3$ over $K_2$ is
   $t^2-111-\al_2$.  If we apply $\phi_{21}$ to the coefficients of this,
   we just get the polynomial $t^2-111-\al_2$ again.  The extensions of
   $\phi_{21}$ over $K_3$ biject with the roots in $\R$ of this
   polynomial, which are $\al_3\simeq 10.847$ and $-\al_3\simeq
   -10.847$.  More explicitly, any element $a\in K_3$ can be written as
   \[ a = a_0 + a_1\al_1 + a_2\al_2 + a_3\al_1\al_2 +
          a_4\al_3 + a_5\al_1\al_3 + a_6\al_2\al_3 + a_7\al_1\al_2\al_3,
   \]
   and we then have
   \begin{align*}
    \phi_{31}(a) &=
          a_0 + a_1\al_1 + a_2\al_2 + a_3\al_1\al_2 +
          a_4\al_3 + a_5\al_1\al_3 + a_6\al_2\al_3 + a_7\al_1\al_2\al_3\\
    \phi_{32}(a) &=
          a_0 + a_1\al_1 + a_2\al_2 + a_3\al_1\al_2 -
          a_4\al_3 - a_5\al_1\al_3 - a_6\al_2\al_3 - a_7\al_1\al_2\al_3.
   \end{align*}
   Now consider instead extensions of the homomorphism $\phi_{22}$.
   These biject with the roots in $\R$ of the polynomial $t^2-111+\al_2$
   (obtained by applying $\phi_{22}$ to the coefficients of
   $\min(\al_3,K_2)(t)=t^2-111-\al_2$).  Here $111-\al_2\simeq
   104.342>0$ so there are two roots, say $\al'_3\simeq 10.214$ and
   $-\al'_3\simeq -10.214$.  This gives two extensions of $\phi_{22}$:
   \begin{align*}
    \phi_{33}(a) &=
          a_0 + a_1\al_1 - a_2\al_2 - a_3\al_1\al_2 +
          a_4\al'_3 + a_5\al_1\al'_3 - a_6\al_2\al'_3 - a_7\al_1\al_2\al'_3\\
    \phi_{34}(a) &=
          a_0 + a_1\al_1 - a_2\al_2 - a_3\al_1\al_2 -
          a_4\al'_3 - a_5\al_1\al'_3 + a_6\al_2\al_3 + a_7\al_1\al_2\al_3.
   \end{align*}
  \item[(d)] We now have
   $E_\Q(K,\R)=\{\phi_{31},\phi_{32},\phi_{33},\phi_{34}\}$ and so
   $|E_\Q(K,\R)|=4$.  On the other hand, we have
   $[K:\Q]=[K_3:K_2][K_2:K_1][K_1:K_0]=2\tm 2\tm 2=8$, so
   $|E_\Q(K,\R)|<[K:\Q]$ as claimed.
  \item[(e)] The same methods show that there are eight different
   homomorphisms from $K$ to $\C$, which can be characterised as
   follows:
   \begin{align*}
    \phi_{31}(\al_3) &= \sqrt{111+\sqrt{11+\sqrt{1111}}} \simeq 10.847 \\
    \phi_{32}(\al_3) &= -\sqrt{111+\sqrt{11+\sqrt{1111}}} \simeq -10.847 \\
    \phi_{33}(\al_3) &= \sqrt{111-\sqrt{11+\sqrt{1111}}} \simeq 10.214 \\
    \phi_{34}(\al_3) &= -\sqrt{111-\sqrt{11+\sqrt{1111}}} \simeq -10.214 \\
    \phi_{35}(\al_3) &= \sqrt{111+\sqrt{11-\sqrt{1111}}} \simeq 10.538+0.224i \\
    \phi_{36}(\al_3) &= -\sqrt{111+\sqrt{11-\sqrt{1111}}} \simeq -10.538-0.224i \\
    \phi_{37}(\al_3) &= \sqrt{111-\sqrt{11-\sqrt{1111}}} \simeq 10.214-0.224i \\
    \phi_{38}(\al_3) &= -\sqrt{111-\sqrt{11-\sqrt{1111}}} \simeq -10.214+0.224i.
   \end{align*}
 \end{itemize}
\EndDeferredSolution

\BeginDeferredSolution{ex-dedekind-direct}{6.4}
 Suppose we have $\sum_ib_i\tht_i$, or in other words
 $\sum_ib_i\tht_i(a)=0$ for all $a\in L$.  Taking $a=1$ we get
 \begin{align*}
  b_0+b_1+b_2+b_3 &= 0 \tag{A} \\
  \intertext{Similarly, we can take $a$ to be $\sqrt{p}$, $\sqrt{q}$
   or $\sqrt{pq}$ to get three more equations:}
  b_0\sqrt{p}+b_1\sqrt{p}-b_2\sqrt{p}-b_3\sqrt{p} &= 0 \\
  b_0\sqrt{q}-b_1\sqrt{q}+b_2\sqrt{q}-b_3\sqrt{q} &= 0 \\
  b_0\sqrt{pq}-b_1\sqrt{pq}-b_2\sqrt{pq}+b_3\sqrt{pq} &= 0. \\
  \intertext{After dividing by $\sqrt{p}$, $\sqrt{q}$ and $\sqrt{pq}$
   respectively we get}
  b_0+b_1-b_2-b_3 &= 0 \tag{B} \\
  b_0-b_1+b_2-b_3 &= 0 \tag{C} \\
  b_0-b_1-b_2+b_3 &= 0. \tag{D}
 \end{align*}
 Adding~(A), (B), (C) and~(D) gives $b_0=0$.  We can then add~(A)
 and~(B) to get $b_1=0$.  Similar manipulations then give $b_2=b_3=0$,
 as required.
\EndDeferredSolution

\BeginDeferredSolution{ex-basis-misc-i}{6.5}
\ \\
 \begin{itemize}
  \item[(a)] Note that $\al$ is a root of the polynomial $f(x)=x^4-2$,
   which is irreducible over $\Q$ by Eisenstein's criterion at the prime $2$.
   It follows that $f(x)$ is the minimal polynomial of $\al$ over
   $\Q$, and so $[\Q(\al):\Q]=\deg(f(x))=4$.
  \item[(b)] Any element of $a\in K$ can be written as $a=x+iy$ with
   $x,y\in\Q(\al)$, and $x$ and $y$ are the real and imaginary part of
   $a$, so they are uniquely determined.  It follows that $1,i$ is a
   basis for $K$ over $\Q(\al)$, so $[K:\Q(\al)]=2$.  We see in the
   same way that $[\Q(i):\Q]=2$.
  \item[(c)] We now have
   \[ [K:\Q(i)][\Q(i):\Q] = [K:\Q] = [K:\Q(\al)][\Q(\al):\Q]. \]
   After inserting the values obtained in~(a) and~(b) we see that
   $[K:\Q]=8$ and $[K:\Q(i)]=4$.
  \item[(d)] We have $f(x)=(x-\al)(x-i\al)(x-i^2\al)(x-i^3\al)$ in
   $K[x]$, so $K$ is a splitting field for $f(x)$ over $\Q(i)$, so it
   is normal over $\Q(i)$.  Note also that
   $[K:\Q(i)]=[\Q(i,\al):\Q(i)]=4$, so $\min(\al,\Q(i))$ must have
   degree $4$, so it must be the same as $f(x)$.  This means that
   $f(x)$ is still irreducible over $\Q(i)$, so the Galois group acts
   transitively on the roots.  Thus, there is an automorphism
   $\sg\in G(K/\Q(i))$ with $\sg(\al)=i\al$.  Alternatively, we can be
   more concrete as follows.  Every element $a\in K$ can be written in
   a  unique way as $a=a_0+a_1\al+a_2\al^2+a_3\al^3$ with
   $a_0,\dotsc,a_3\in\Q(i)$.  We can thus define a $\Q(i)$-linear map
   $\sg\:K\to K$ by
   \[ \sg(a_0+a_1\al+a_2\al^2+a_3\al^3) =
        a_0+ia_1\al+i^2a_2\al^2+i^3a_3\al^3.
   \]
   It is clear that $\sg$ respects addition and sends $0$ to $0$ and
   $1$ to $1$.  Just by expanding everything out, one can also check
   that $\sg(ab)=\sg(a)\sg(b)$, so $\sg$ is a homomorphism.  We now
   find that $1,\sg,\sg^2$ and $\sg^3$ are all different, but that
   $\sg^4=1$.  Thus $\sg$ generates a subgroup of $G(K/\Q(i))$
   isomorphic to $C_4$.  As $|G(K/\Q(i))|=[K:\Q(i)]=4$, this must be
   the whole group.
 \end{itemize}
\EndDeferredSolution

\BeginDeferredSolution{ex-which-normal-cyclic}{6.6}
\ \\
 \begin{itemize}
  \item[(a)] Here $L=\Q(\mu_5)$, so we know from the general
   cyclotomic theory that $L$ is Galois over $\Q$, and the Galois
   group is $(\Z/5\Z)^\tm=\{\ov{-2},\ov{-1},\ov{1},\ov{2}\}$.  As
   $\Z/5\Z$ is a field we know that $(\Z/5\Z)^\tm$ is cyclic.
   Explicitly, we have $\ov{2}^2=\ov{4}=\ov{-1}$, and it follows
   easily from this that the group is generated by $\ov{2}$.

  \item[(b)] Here $K$ and $L$ are both normal over $\Q$, and
   $G(L/\Q)=(\Z/25\Z)^{\tm}$ whereas $G(K/\Q)=(\Z/5\Z)^\tm$.  More
   explicitly, we can put $\zt=e^{2\pi i/25}$, and for each
   $k\in(\Z/25\Z)^\tm$ there is a unique automorphism $\sg_k$ of $L$
   with $\sg_k(\zt)=\zt^k$.  Note that $K=\Q(\zt^5)$, so $\sg_k$ acts
   as the identity on $K$ if and only if $\zt^{5k}=\zt^5$, or
   equivalently $5k=5\pmod{25}$, or equivalently $k=1\pmod{5}$.  This
   means that
   \[ G(L/K) = \{\sg_1,\sg_6,\sg_{11},\sg_{16},\sg_{21}\} =
       \{\sg_{1+5i}\st 0\leq i<5\}.
   \]
   Note that $\sg_i$ only depends on $i$ modulo $25$, so
   \[ \sg_{1+5i}\sg_{1+5j} = \sg_{1+5i+5j+25ij} = \sg_{1+5(i+j)}. \]
   It follows from this that $G(L/K)$ is cyclic of order $5$,
   generated by $\sg_6$.

  \item[(c)] Here the polynomial $f(x)=x^5-12$ is irreducible over
   $\Q$ (by Eisenstein's criterion at the prime $3$) and has a root in
   $L$.  However, we have $L\sse\R$ and $f(x)$ has only one real root
   so $f(x)$ does not split in $L[x]$.  It follows that $L$ is not
   normal over $K$.

  \item[(c)] This is normal, with Galois group $C_5$.  Here is a
   rigorous argument (in practice, you wouldn't necessarily write down
   all these steps):

   Firstly, observe that $[L:\Q]=20$. For we have
   \[ [L:\Q]=[L:K][K:\Q] \]
   and
   \[ [L:\Q]=[L:\Q(\sqrt[5]{3})][\Q(\sqrt[5]{3}):\Q]. \]
   As $[K:\Q]=4$ (by (a)) and $[\Q(\sqrt[5]{3}):\Q]=5$ (the minimal
   polynomial is $x^5-3$, irreducible by Eisenstein with $p=3$), we
   see that $[L:\Q]$ is a multiple of 4 and of 5, so is divisible by
   20. Conversely, $[L:K]=[K(\sqrt[5]{3}):K]\leq 5$, as it is the degree
   of the minimal polynomial of $\sqrt[5]{3}$ over $K$, and this must
   divide $x^5-3$, so be of degree at most 5. As $[L:\Q]=[L:K][K:\Q]$,
   we see $[L:\Q]\leq 20$. Combining these, we get that $[L:\Q]=20$ and
   thus that $[L:K]=5$.

   So $x^5-3$ is the minimal polynomial of $\sqrt[5]{3}$ over
   $K$. Write $\al=\sqrt[5]{3}$ and $\zt=e^{{2\pi i}/{5}}$. The roots
   of the minimal polynomial are $\al$, $\al\zt$, $\al\zt^2$,
   $\al\zt^3$ and $\al\zt^4$. All these roots lie in $K(\al)$, so it
   follows that $|G(K(\al)/K)|=5$, and the extension is Galois.

   As every group with 5 elements is cyclic, this implies that the
   Galois group is $C_5$. Explicitly, however, the 5 automorphisms are
   determined by the their effects on $\al$; $\al$ must be sent to one
   of $\al$, $\al\zt$, $\al\zt^2$, $\al\zt^3$ or $\al\zt^4$. It is
   easy to see that the automorphism sending $\al$ to $\al\zt$
   generates all of the automorphisms (as do any of the non-trivial
   automorphisms).
 \end{itemize}
\EndDeferredSolution

\BeginDeferredSolution{ex-two-roots-basis}{7.1}
 The obvious basis is the set $B=\{1,\sqrt{2},\sqrt{3},\sqrt{6}\}$.
 Note that
 \[ \frac{1}{2+\sqrt{2}+\sqrt{3}} =
    \frac{2+\sqrt{2}-\sqrt{3}}{(2+\sqrt{2}-\sqrt{3})(2+\sqrt{2}+\sqrt{3})} =
    \frac{2+\sqrt{2}-\sqrt{3}}{(2+\sqrt{2})^2-(\sqrt{3})^2} =
    \frac{2+\sqrt{2}-\sqrt{3}}{3+4\sqrt{2}}.
 \]
 Here
 \[ \frac{1}{3+4\sqrt{2}} =
    \frac{3-4\sqrt{2}}{(3+4\sqrt{2})(3-4\sqrt{2})} =
    \frac{3-4\sqrt{2}}{3^2-(4\sqrt{2})^2} =
    \frac{4\sqrt{2}-3}{23}.
 \]
 Putting this together, we get
 \[ \frac{1}{2+\sqrt{2}+\sqrt{3}} =
    (2+\sqrt{2}-\sqrt{3})(4\sqrt{2}-3)/23 =
    \tfrac{2}{23} + \tfrac{5}{23}\sqrt{2} +
     \tfrac{3}{23}\sqrt{3} - \tfrac{4}{23}\sqrt{6}.
 \]
\EndDeferredSolution

\BeginDeferredSolution{ex-three-five}{7.2}
 Clearly $\Q(\sqrt{3}+\sqrt{5})\subseteq\Q(\sqrt{3},\sqrt{5})$.  But
 if $\al=\sqrt{3}+\sqrt{5}$, then $\al^3=18\sqrt{3}+14\sqrt{5}$,
 so
 \begin{align*}
  \sqrt{3} &= \frac{\al^3-14\al}{4} \\
  \sqrt{5} &= \frac{18\al - \al^3}{4}.
 \end{align*}
 This gives the other inclusion.
\EndDeferredSolution

\BeginDeferredSolution{ex-biquadratic}{7.3}
 Put $\al=\sqrt{p}+\sqrt{q}\in\Q(\sqrt{p},\sqrt{q})$.  Then
 \begin{align*}
  \al^2    &= p+q+2\sqrt{pq} &
  \al^3    &= (p+3q)\sqrt{p}+(q+3p)\sqrt{q},
 \end{align*}
 so
 \[ \sqrt{p} = \frac{\al^3-(q+3p)\al}{2(q-p)}
    \hspace{4em}
    \sqrt{q} = \frac{\al^3-(p+3q)\al}{2(p-q)}.
 \]
 This shows that $\sqrt{p},\sqrt{q}\in\Q(\al)$,
 $\Q(\al)=\Q(\sqrt{p},\sqrt{q})$.  The assumed linear independence
 statement shows that $[\Q(\sqrt{p},\sqrt{q}):\Q]=4$, so
 $[\Q(\al):\Q]=4$, so the minimal polynomial $\min(\al,\Q)$ must have
 degree $4$.  We saw above that $\al^2=p+q+2\sqrt{pq}$, so
 $(\al^2-(p+q))^2=4pq$, so $\al^4-2(p+q)\al+(p+q)^2-4pq=0$.  As
 $(p+q)^2-4pq=(p-q)^2$, this can be rewritten as $f(\al)=0$.  This
 means that $f(x)$ is divisible by $\min(\al,\Q)$, but both these
 polynomials are monic of degree $4$, so they must be the same.
 One can show in the same way that $f(\pm\sqrt{p}\pm\sqrt{q})=0$, for
 any of the four possible choices of signs.  Alternatively, we can
 perform the following expansion:
 \begin{align*}
   & (x-\sqrt{p}-\sqrt{q})
     (x-\sqrt{p}+\sqrt{q})
     (x+\sqrt{p}-\sqrt{q})
     (x+\sqrt{p}+\sqrt{q}) \\
  =& ((x-\sqrt{p})^2-q)((x+\sqrt{p})^2-q) =
     (x^2-2\sqrt{p}x+p-q)(x^2+2\sqrt{p}x+p-q) \\
  =& (x^2+p-q)^2 - (2\sqrt{p}x)^2
      = x^4 - 2(p+q)x^2 + (p-q)^2 = f(x).
 \end{align*}
 Either way, we see that the roots of $f(x)$ are $\sqrt{p}+\sqrt{q}$,
 $\sqrt{p}-\sqrt{q}$, $-\sqrt{p}+\sqrt{q}$ and $-\sqrt{p}-\sqrt{q}$,
 so the splitting field of $f(x)$ is $\Q(\sqrt{p},\sqrt{q})$.

 On the other hand, we see by inspection that
 \[ g(x)=(x^2-p)(x^2-q)=
     (x-\sqrt{p})(x+\sqrt{p})(x-\sqrt{q})(x+\sqrt{q}).
 \]
 It is clear from this that the splitting field of $g(x)$ is also
 $\Q(\sqrt{p},\sqrt{q})$.
\EndDeferredSolution

\BeginDeferredSolution{ex-galois-i}{7.4}
 Put $\al=\sqrt[3]{3}\in\R$ and $\om=e^{2\pi i/3}=(\sqrt{-3}-1)/2$, so
 $L$ can also be described as $\Q(\al,\om)$.  Put
 $f(t)=t^3-3\in\Q[t]$.  This is irreducible over $\Q$ by Eisenstein's
 criterion at the prime $3$, but it splits over $L$ as
 $(t-\al)(t-\om\al)(t-\om^2\al)$.  It follows that $L$ is the
 splitting field of $f(t)$, so that the Galois group $G=G(L/\Q)$ can
 be regarded as a group of permutations of the set
 $R=\{\al,\om\al,\om^2\al\}$.  This group acts transitively on $R$
 (because $f(t)$ is irreducible), so it must be either the full group
 $\Sg_R$ of all permutations, or the subgroup $A_R$ of even
 permutations.  However, complex conjugation restricts to give an
 automorphism of $L$ corresponding to the transpositon that exchanges
 $\om\al$ and $\om^2\al$.  This shows that $G(L/K)\not\sse A_R$, so we
 must have $G(L/K)=\Sg_R\simeq\Sg_3$.
\EndDeferredSolution

\BeginDeferredSolution{ex-galois-ii}{7.5}
 There is an automorphism $\sg$ of $L$ given by $z\mapsto\ov{z}$.  We
 claim that this is the only nontrivial automorphism.  To see this,
 write $\al=\sqrt[3]{3}$, so $L=\Q(\al,i)$ and
 \[ L\cap\R = \Q(\al) = \{a+b\al+c\al^2\st a,b,c\in\Q\}. \]
 We will need to know that $\sqrt{3}$ does not lie in $L$.  It
 certainly does not appear to lie in $L$, but there could in principle
 be a strange coincidence, so we should check rigorously.  As
 $\sqrt{3}$ is real, if it lay in $L$ we would have
 $\sqrt{3}=a+b\al+c\al^2$ for some $a,b,c\in\Q$.  Squaring this gives
 \[ (a^2+6bc) + (2ab+3c^2)\al + (2ac+b^2)\al^2 = 3, \]
 so
 \begin{align*}
  a^2+6bc  &= 3 \\
  2ab+3c^2 &= 0 \\
  2ac+b^2  &= 0.
 \end{align*}
 If either of $b$ or $c$ is zero then the first equation gives
 $a^2=3$, which is impossible as $a$ is rational.  We may thus assume
 that $b$ and $c$ are nonzero, and rearrange the second and third
 equations as $3c^2/b=-2a=b^2/c$, and thus $3=(b/c)^3$.  This is again
 impossible, as $b/c$ is rational.  Thus, we have $\sqrt{3}\not\in L$,
 as expected.  Now consider $\om=e^{2\pi i/3}=(\sqrt{3}i-1)/2$.  If
 this were in $L$, then $(2\om+1)/i=\sqrt{3}$ would also be in $L$,
 which is false.  So $\om\not\in L$, and similarly
 $\om^{-1}\not\in L$, so the only cube root of unity in $L$ is $1$.

 Now let $\rho$ be any automorphism of $L$.  Then
 $\rho(i)^2+1=\rho(i^2+1)=\rho(0)=0$, so $\rho(i)=\pm i$.  Similarly
 $(\rho(\al)/\al)^3=\rho(\al^3)/\al^3=\rho(3)/3=1$, so $\rho(\al)/\al$
 is a cube root of unity in $L$.  By the previous paragraph we
 therefore have $\rho(\al)=\al$.  It follows that $\rho$ is either the
 identity (if $\rho(i)=i$) or $\sg$ (if $\rho(i)=-i$).

 As $1$ and $\sg$ both act as the identity on $\al$, we see that
 $G(L/\Q(\al))=G(L/\Q)=\{1,\sg\}$.  Now
 $[L:\Q(\al)]=2=|G(L/\Q(\al))|$, so $L$ is normal over $\Q(\al)$.  On
 the other hand, $[L:\Q]=4>2=|G(L/\Q)|$, so $L$ is not normal over
 $\Q$.  Explicitly, the polynomial $f(t)=t^3-3\in\Q[t]$ has a root in
 $L$ but does not split in $L$.
\EndDeferredSolution

\BeginDeferredSolution{ex-galois-iii}{7.6}
 Put $\al=\sqrt[4]{3}$ and
 \[ f(t) = (t-\al)(t+\al)(t-i\al)(t+i\al). \]
 We find that $(t-\al)(t+\al)=t^2-\sqrt{3}$, but
 $(t-i\al)(t+i\al)=t^2+\sqrt{3}$, so $f(t)=t^4-3$.  It follows easily
 that $L=\Q(\al,i)$ is a splitting field for $f(t)$ over $\Q$, so $L$
 is normal over $\Q$.  The set $R=\{\al,i\al,-\al,-i\al\}$ of roots is
 the set of vertices of a square in the complex plane.  We claim that
 the group $G(L/\Q)$ is just the dihedral group of rotations and
 reflections of this square.  Indeed, complex conjugation gives an
 automorphism $\sg$ which reflects the square across the real axis.
 Next, we can use Eisenstein's criterion at the prime $3$ to see that
 $f(t)$ is irreducible, so $G(L/\Q)$ acts transitively on $R$.  It
 follows that there is an automorphism $\phi$ with $\phi(\al)=i\al$.
 Now $\phi(i)$ must be a square root of $-1$, so $\phi(i)=\pm i$.  If
 $\phi(i)=i$ then we put $\rho=\phi$, otherwise we put
 $\rho=\phi\sg$.  Either way we find that $\rho(i)=i$ and
 $\rho(\al)=i\al$.  This implies that $\rho(i^m\al)=i^{m+1}\al$ for
 all $m$, so $\rho$ is a quarter turn of the square.  This means that
 $\rho$ and $\sg$ generate $D_8$, so $|G(L/\Q)|\geq|D_8|=8$.  On the
 other hand, the set
 \[ B = \{1,\al,\al^2,\al^3,i,i\al,i\al^2,i\al^3\} \]
 clearly spans $L$ over $\Q$, so $[L:\Q]\leq |B|=8$, and for any
 extension we have $|G(L/\Q)|\leq [L:\Q]$.  It follows that all these
 inequalities must be equalities, so $G(L/\Q)=D_8$ and $B$ is a basis.
 \begin{center}
  \begin{tikzpicture}[scale=1.5]
   \draw[blue] (1.32,0) -- (0,1.32) -- (-1.32,0) -- (0,-1.32) -- cycle;
   \fill ( 0.00, 0.00) circle(0.05);
   \fill ( 1.32, 0.00) circle(0.05);
   \fill ( 0.00, 1.32) circle(0.05);
   \fill (-1.32, 0.00) circle(0.05);
   \fill ( 0.00,-1.32) circle(0.05);
   \draw ( 1.54, 0.00) node{$\al$};
   \draw ( 0.00, 1.54) node{$i\al$};
   \draw (-1.59, 0.00) node{$-\al$};
   \draw ( 0.00,-1.54) node{$-i\al$};
   \draw[red,ultra thick,<->] (1.7,-0.3) -- (1.7,0.3);
   \draw[red,ultra thick,->] (0,0) +(0:0.3) arc(0:90:0.3);
   \draw[red] (1.85,0) node{$\sg$};
   \draw[red] (0.4,0.4) node{$\rho$};
  \end{tikzpicture}
 \end{center}
\EndDeferredSolution

\BeginDeferredSolution{ex-galois-iv}{7.7}
 We will do~(a) and~(b) first, and then check that $f(x)$ is
 irreducible.
 \begin{itemize}
  \item[(a)] From the definition we have $2\al^2+1=\sqrt{-15}$, and
   squaring again gives $4\al^4+4\al^2+16=0$, so $f(\al)=0$.  As
   $f(x)$ only involves even powers of $x$ we have $f(-x)=f(x)$ and so
   $f(-\al)=0$.  Now
   \[ f(2/\al) = \frac{16}{\al^4} + \frac{4}{\al^2} + 4 =
       \frac{4}{\al^4}(4+\al^2+\al^4) = \frac{4}{\al^4}f(\al)= 0,
   \]
   and similarly $f(-2/\al)=0$.  Numerically we have
   $\al\simeq 0.87+0.12i$, and from that one can check that
   $\al,-\al,2/\al$ and $-2/al$ are all distinct.  We must therefore
   have
   \[ f(x) = (x-\al)(x+\al)(x-2/\al)(x+2/\al). \]

  \item[(b)] We have a normal extension of degree $4$, so the Galois
   group $G$ must have order $4$.  We know that $G$ acts transitively
   on the roots, so there are automorphisms $\sg$ and $\rho$ with
   $\sg(\al)=-\al$ and $\rho(\al)=2/\al$.  These satisfy
   $\sg^2(\al)=\sg(-\al)=-\sg(\al)=\al$ and
   $rho^2(\al)=\rho(2/\al)=2/\rho(\al)=\al$, so $\sg^2=\rho^2=1$.  We
   also have $\sg(\rho(\al))=\rho(\sg(\al))=-2/\al$.  It follows that
   \[ G = \{1,\sg,\rho,\sg\rho\}, \]
   and this is isomorphic to $C_2\tm C_2$.
 \end{itemize}

 We now prove that $f(x)$ is irreducible.  It is clear that $f(x)>0$
 for all $x\in\R$, so there are no roots in $\Q$.  This means that the
 only way $f(x)$ could factor would be as the product of two
 quadratics, say $f(x)=(x^2+ax+b)(x^2+cx+d)$ for some $a,b,c,d\in\Q$.
 By looking at the term in $x^3$, we see that $c=-a$.  After
 substituting this, expanding and comparing the remaining coefficients
 we obtain
 \begin{align*}
  b+d-a^2 &= 1 \\
  a(d-b) &= 0 \\
  bd &= 4.
 \end{align*}
 If $a=0$ we quickly obtain $b=(1\pm\sqrt{-3})/2$, which is impossible
 as $b\in\Q$.  Thus $a\neq 0$, so the second equation above gives
 $d=b$, so the last equation gives $b=\pm 2$.  The first equation then
 becomes $a^2=\pm 4-1$, which is impossible for $a\in\Q$.
\EndDeferredSolution

\BeginDeferredSolution{ex-galois-v}{7.8}
\ \\
 \begin{itemize}
  \item[(a)] As $f(x)=x^4\pmod{2}$ and $f(0)\neq 0\pmod{4}$ we can use
   Eisenstein's criterion to see that $f(x)$ is irreducible.
  \item[(b)] Note that $\al^2+4=3\sqrt{2}=\sqrt{18}$, and squaring
   again shows that $\al^4+8\al^2+16=18$, so $f(\al)=0$.  As $f(x)$
   only involves even powers of $x$ we have $f(-x)=f(x)$ and so
   $f(-\al)=0$.  Now put $\bt=\sqrt{-3\sqrt{2}-4}$; the same argument
   shows that $f(\pm\bt)=0$.  We also have
   $(\al\bt)^2=(3\sqrt{2}-4)(-3\sqrt{2}-4)=-2$, so
   $\bt=\pm\sqrt{-2}/\al$.  (With the standard conventions for square
   roots we have $\al>0$, and $\bt$ and $\sqrt{-2}$ are positive
   multiples of $i$, and it follows that $\bt=\sqrt{-2}/\al$.)  It
   follows that the roots of $f(x)$ are as described, so the splitting
   field is $\Q(\al,\bt)=\Q(\al,\al\bt)=\Q(\al,\sqrt{-2})=M$ as
   claimed.
  \item[(c)] We have $3\sqrt{2}-4\simeq 0.24>0$ so $\al$ is real, so
   $\Q(\al)\sse M\cap\R$.  As $f(x)$ is irreducible, it must be the
   minimal polynomial for $\al$, and so $[\Q(\al):\Q]=\deg(f(x))=4$.
   As $\Q(\al)\sse\R$ and $\sqrt{-2}$ is purely imaginary we see that
   $1,\sqrt{-2}$ is a basis for $M$ over $\Q(\al)$, so
   $M\cap\R=\Q(\al)$ and $[M:\Q]=[M:\Q(\al)][\Q(\al):\Q]=2\tm 4=8$.
  \item[(d)] First let $\psi\:M\to M$ be given by complex conjugation,
   so $\psi(\sqrt{-2})=-\sqrt{-2}$ and $\psi(\al)=\al$.  It is clear
   that $\psi^2=1$.  Next, the Galois group of the splitting field of
   an irreducible polynomial always acts transitively on the roots, so
   we can find $\sg\in G(M/\Q)$ with $\sg(\al)=\sqrt{-2}/\al$.  Now
   $\sg$ must permute the roots of $x^2+2$, so
   $\sg(\sqrt{-2})=\pm\sqrt{-2}$.  If the sign is positive we put
   $\phi=\sg\psi$, otherwise we put $\phi=\sg$.  In either case we
   then have $\phi(\al)=\sqrt{-2}/\al=\bt$ and
   $\phi(\sqrt{-2})=-\sqrt{-2}$.  This means that
   \[ \phi^2(\al)=\phi(\sqrt{-2}/\al)=\phi(\sqrt{-2})/\phi(\al) =
       -\sqrt{-2}/(\sqrt{-2}/\al) = -\al
   \]
   and $\phi^2(\sqrt{-2})=\sqrt{-2}$.  It follows in turn that
   $\phi^4=1$.  We now have various different automorphisms, whose
   effect we can tabulate as follows:
   \[ \renewcommand{\arraystretch}{1.5}
      \begin{array}{|c||c|c|c|c|c|c|c|c|} \hline
       & 1 & \phi & \phi^2 & \phi^3 &
         \psi & \phi\psi & \phi^2\psi & \phi^3\psi \\ \hline
       \al & \al & \bt & -\al & -\bt & \al & \bt & -\al & -\bt \\ \hline
       \bt & \bt & -\al & -\bt & \al & -\bt & \al & \bt & -\al \\ \hline
       \sqrt{-2} & \sqrt{-2} & -\sqrt{-2} & \sqrt{-2} & -\sqrt{-2} &
                   -\sqrt{-2} & \sqrt{-2} & -\sqrt{-2} & \sqrt{-2}. \\
       \hline
      \end{array}
   \]
   We see that the eight automorphisms listed are all different, but
   $|G(M/\Q)|=[M:\Q]=8$, so we have found all the automorphisms.
  \item[(e)] We can read off from the above table that
   $\psi\phi\psi^{-1}=\phi^3=\phi^{-1}$.  This means that $G(M/\Q)$ is
   the dihedral group $D_8$, with $\phi$ corresponding to a rotation
   through $\pi/2$, and $\psi$ to a reflection.
 \end{itemize}
\EndDeferredSolution

\BeginDeferredSolution{ex-cyclotomic-twenty}{8.1}
 Recall the key fact that
 \[ x^n-1 = \prod_{d|n} \vph_d(x). \]
 In particular, we have
 \begin{align*}
  x-1 &= \vph_1(x) \\
  x^2-1 &= \vph_1(x)\vph_2(x) \\
  x^4-1 &= \vph_1(x)\vph_2(x)\vph_4(x) \\
  x^5-1 &= \vph_1(x)\vph_5(x) \\
  x^{10}-1 &= \vph_1(x)\vph_{2}(x)\vph_5(x)\vph_{10}(x) \\
  x^{20}-1 &= \vph_1(x)\vph_{2}(x)\vph_4(x)
              \vph_5(x)\vph_{10}(x)\vph_{20}(x).
 \end{align*}
 Dividing the second and third of these gives
 \[ \vph_4(x) = \frac{x^4-1}{x^2-1} = x^2+1. \]
 On the other hand, we can divide the last two equations to give
 \[ \vph_{20}(x)\vph_4(x) =
     \frac{x^{20}-1}{x^{10}-1} = x^{10}+1.
 \]
 Putting these together, we get
 \[ \vph_{20}(x) =
     \frac{x^{10}+1}{x^2+1} = x^8-x^6+x^4-x^2+1.
 \]
 (The calculation can also be arranged in various other ways, but this
 is probably the most efficient.)
\EndDeferredSolution

\BeginDeferredSolution{ex-phi-CC}{8.2}
 We have
 \begin{align*}
  x^{200}-1 &=
   \vph_{200}(x)\vph_{100}(x)\vph_{50}(x)\vph_{40}(x)
   \vph_{25}(x)\vph_{20}(x)\vph_{10}(x)\vph_8(x)\vph_5(x)
   \vph_4(x)\vph_2(x)\vph_1(x)\\
  x^{100}-1 &=
   \vph_{100}(x)\vph_{50}(x)\vph_{25}(x)\vph_{20}(x)
   \vph_{10}(x)\vph_5(x)\vph_4(x)\vph_2(x)\vph_1(x)\\
  x^{40}-1  &=
   \vph_{40}(x)\vph_{20}(x)\vph_{10}(x)\vph_8(x)
   \vph_5(x)\vph_4(x)\vph_2(x)\vph_1(x)\\
  x^{20}-1  &=
   \vph_{20}(x)\vph_{10}(x)\vph_5(x)\vph_4(x)\vph_2(x)\vph_1(x)
 \end{align*}
 and it follows that
 \[ \vph_{200}(x) =
     \frac{(x^{200}-1)(x^{20}-1)}{(x^{100}-1)(x^{40}-1)} =
     \frac{x^{100}+1}{x^{20}+1} =
      x^{80}-x^{60}+x^{40}-x^{20}+1.
 \]
\EndDeferredSolution

\BeginDeferredSolution{ex-mu-seven}{8.3}
 Put $\zt=e^{3\pi i/7}=(e^{2\pi i/14})^3$ and
 $\al=\zt+1$.  As $3$ and $14$ are coprime, we see that $\zt$ is a
 primitive 14th root of unity, and so is a root of the cyclotomic
 polynomial $\vph_{14}(t)$.  We know that
 \begin{align*}
  t^{14}-1 &= \vph_{14}(t)\vph_7(t)\vph_2(t)\vph_1(t) \\
  t^7 - 1  &= \vph_7(t)\vph_1(t) \\
  t+1      &= \vph_2(t).
 \end{align*}
 We can divide the first of these by the second and the third to give
 \[ \vph_{14}(t) =
    \frac{t^7+1}{t+1} = t^6-t^5+t^4-t^3+t^2-t+1.
 \]
 Now put $f(t)=\vph_{14}(t-1)$.  This is again a polynomial of degree
 $6$ over $\Q$, and we have
 $f(\al)=\vph_{14}(\al-1)=\vph_{14}(\zt)=0$.  More explicitly, we can
 use the expression $\vph_{14}(t)=(t^7+1)/(t+1)$ to get
 \[ f(t) = \frac{(t-1)^7+1}{t-1+1} =
     ((t-1)^7+1)/t = \sum_{i=0}^6 (-1)^i\bcf{7}{i} t^{6-i} =
      t^6-7t^5+21t^4-35t^3+35t^2-21t+7.
 \]
 This reduces to $t^6$ modulo $7$, either by inspecting the
 coefficients directly, or by recalling that
 $(t-1)^7=t^7-1^7\pmod{7}$.  Moreover, the constant term is $7$, which
 is not divisible by $7^2$.  Thus Eisenstein's criterion is
 applicable, and we see that $f(t)$ is irreducible.
\EndDeferredSolution

\BeginDeferredSolution{ex-mu-fifteen}{8.4}
 Put $\zt=e^{2\pi i/15}$ and $K=\Q(\zt)=\Q(\mu_{15})$.  The general
 theory tells us that for each integer $k$ that is coprime to $15$,
 there is a unique automorphism $\sg_k$ of $K$ with
 $\sg_k(\zt)=\zt^k$, and that the rule $k+15\Z\mapsto\sg_k$ gives a
 well-defined isomorphism $(\Z/15\Z)^\tm\to G(K/\Q)$.  Every element
 of $\Z/15\Z$ has a unique representative lying between $-7$ and $7$,
 and the integers in that range that are coprime to $15$ form the set
 \[ U = \{-7,-4,-2,-1,1,2,4,7\}, \]
 so we can identify this set with $(\Z/15\Z)^\tm$.  Put $A=\{1,-1\}$,
 which is a cyclic subgroup of $U$ of order $2$.  Note that
 $2^3=8=-7\pmod{15}$ and $2^4=16=1\pmod{15}$.  It follows that the set
 $B=\{1,2,4,-7\}$ is a cyclic subgroup of $U$ of order $4$, and we see
 directly that $U=A\tm B$.
\EndDeferredSolution

\BeginDeferredSolution{ex-cyclotomic-real}{8.5}
\ \\
 \begin{itemize}
  \item[(a)] Put $f(x)=x^2-\bt x+1\in\Q(\bt)[x]$.  As
   $\bt=\zt+\zt^{-1}$, we see that $\bt\zt=\zt^2+1$, so $f(\zt)=0$.
   Thus, $\zt$ satisfies a quadratic equation over $\Q(\bt)$, as
   claimed.  The minimal polynomial $\min(\zt,\Q(\bt))$ must divide
   $f(x)$, so it has degree one (if $\zt\in\Q(\bt)$) or two (if
   $\zt\not\in\Q(\bt)$).  Thus, we have $[\Q(\zt):\Q(\bt)]\leq 2$.

  \item[(b)] We next observe that $\zt^n=1$ so $|\zt|>0$ and
   $|\zt|^n=1$, so $|\zt|=1$.  If $\zt$ is real this means that
   $\zt=\pm 1$, so $\zt^2=1$, but this contradicts the assumption that
   $\zt$ is a primitive $n$th root for some $n\geq 3$.  Thus, we see
   that $\zt\not\in\R$.  On the other hand, as $|\zt|=1$ we see that
   $\zt^{-1}=\ov{\zt}$, so $\bt=\zt+\ov{\zt}=2\text{Re}(\zt)\in\R$.
   It follows that $\Q(\bt)\sse\R$ and so $\zt\not\in\Q(\bt)$.  In
   conjunction with~(a) this means that $[\Q(\zt):\Q(\bt)]=2$.

  \item[(c)] We claim that $\zt^m+\zt^{-m}=p_m(\bt)$ for some
   polynomial $p_m(x)$.  Indeed, we can put $p_0(x)=2$ and $p_1(x)=x$,
   and then define $p_m(x)$ recursively for $m>1$ by
   $p_{k+1}(x)=x\,p_k(x)-p_{k-1}(x)$.  We claim that
   $p_k(\bt)=\zt^k+\zt^{-k}$.  This is clear for $k\in\{0,1\}$.  If
   the claim holds for all $k\leq m$, we have
   \begin{align*}
    p_{m+1}(\bt) &=
     \bt p_m(\bt) - p_{m-1}(\bt) \\
     &= (\zt+\zt^{-1})(\zt^m+\zt^{-m}) - (\zt^{m-1}+\zt^{1-m}) \\
     &= (\zt^{m+1}+\zt^{1-m}+\zt^{m-1}+\zt^{-m-1}) -
         (\zt^{m-1}+\zt^{1-m}) \\
     &= \zt^{m+1}+\zt^{-m-1}.
   \end{align*}
   The claim therefore holds for all $m$, by induction.

  \item[(d)] The first few steps of the recursive scheme are as
   follows:
   \begin{align*}
    p_0(x) &= 2 \\
    p_1(x) &= x \\
    p_2(x) &= x\,p_1(x) - p_0(x) = x^2-2 \\
    p_3(x) &= x\,p_2(x) - p_1(x) = x^3-3x \\
    p_4(x) &= x\,p_3(x) - p_2(x) = x^4-4x^2+2 \\
    p_5(x) &= x\,p_4(x) - p_3(x) = x^5-5x^3+5x.
   \end{align*}
   Thus, we have $\zt^5+\zt^{-5}=\bt^5-5\bt^3+5\bt$.
 \end{itemize}
\EndDeferredSolution

\BeginDeferredSolution{ex-shift-irr}{8.6}
 Suppose that $g(t)=f(t+a)$ is irreducible as above.  Suppose we have
 a factorisation $f(t)=p(t)q(t)$, where $p(t)$ and $q(t)$ are
 nonconstant polynomials in $K[t]$.  We then have nonconstant
 polynomials $r(t)=p(t+a)$ and $s(t)=q(t+a)$ with $g(t)=r(t)s(t)$.
 This is impossible, because $g(t)$ is assumed to be irreducible.
 This means that no such factorisation $f(t)=p(t)q(t)$ can exist, so
 $f(t)$ must be irreducible.

 Now take $f(t)=\vph_p(t)=(t^p-1)/(t-1)$ and $a=1$.  We then have
 \[ g(t) = \frac{(t+1)^p-1}{(t+1)-1} =
     t^{-1}((t+1)^p-1) = \sum_{i=0}^{p-1}\bcf{p}{i+1}t^i.
 \]
 This is monic, and using Lemma~\ref{lem-F-additive} we see that
 $g(t)=t^{p-1}\pmod{p}$, so the coefficients of $t^0,\dotsc,t^{p-2}$
 are all divisible by $p$.  Moreover, the constant term is $g(0)=p$,
 which is not divisible by $p^2$.  Eisenstein's criterion therefore
 tells us that $g(t)=f(t+1)$ is irreducible, so we can use the first
 paragraph above to see that $f(t)$ is also irreducible.
\EndDeferredSolution

\BeginDeferredSolution{ex-phi-two-power}{8.7}
 Put $s=t^{2^k}$.  As the divisors of $2^k$ are just the powers $2^j$
 for $j\leq k$, we have $s-1=\prod_{j=0}^k\vph_{2^j}(t)$.  We also
 have $s^2=t^{2\tm 2^k}=t^{2^{k+1}}$, so
 $s^2-1=\prod_{j=0}^{k+1}\vph_{2^j}(t)$.  By dividing these two
 equations we get $\vph_{2^{k+1}}(t)=(s^2-1)/(s-1)=s+1=t^{2^k}+1$ as
 claimed.

 Alternatively, if $\zt$ is a $2^{k+1}$th root of unity, then
 $\zt^{2^k}$ cannot be equal to $1$ (by primitivity) but
 $(\zt^{2^k})^2=\zt^{2^{k+1}}=1$.  We must therefore have
 $\zt^{2^k}=-1$.  It follows that the primitive $2^{k+1}$th roots of
 unity are precisely the same as the roots of $t^{2^k}+1$.  This
 polynomial is monic and coprime with its derivative, so there are no
 repeated roots.  It follows that $t^{2^k}+1$ is the product of
 $t-\zt$ as $\zt$ runs over the roots, which is $\vph_{2^{n+1}}(t)$.
\EndDeferredSolution

\BeginDeferredSolution{ex-phi-families}{8.8}
 We will write $\mu_k$ for the set of all $k$th roots of unity, and
 $\mu_k^\tm$ for the subset of primitive roots.
 \begin{itemize}
  \item[(a)] Note that $\zt^k=1$ if and only if $\ov{\zt}^k=1$, so
   $\zt$ and $\ov{\zt}$ have the same order.  In other words, $\zt$ is
   a primitive $m$th root of unity if and only if $\ov{\zt}$ is a
   primitive $m$th root of unity.  Now suppose that $m>2$.  The only
   roots of unity on the real axis are $+1$ (of order $1$) and $-1$
   (of order $2$), so all primitive $m$th roots of unity have nonzero
   imaginary part.  Our first observation shows that the roots with
   positive imaginary part biject with those of negative imaginary
   part, so the total number of roots is even.  This number is the
   same as the degree of $\vph_m(x)$.
  \item[(b)] We can write $n=2m$, where $m$ is odd.  Suppose that
   $\zt\in\mu_n^\tm$, so $\zt^k=1$ if and only if $n|k$.  This means
   that $\zt^m\neq 1$, but $(\zt^m)^2=\zt^n=1$, so we must have
   $\zt^m=-1$.  This means that $(-\zt)^m=(-1)^m\zt^m=(-1)^{m+1}$,
   which is $1$ because $m$ is odd.  On the other hand, if
   $(-\zt)^k=1$ then $\zt^{2k}=(-\zt)^{2k}=1^2=1$, so $2k$ must be
   divisible by $n=2m$, so $k$ must be divisible by $m$.  This proves
   that $-\zt\in\mu_m^\tm$.

   Conversely, suppose that $-\zt\in\mu_m^\tm$.  As $m$ is odd we then
   have $\zt^m=(-1)^m(-\zt)^m=-1$, and thus $\zt^n=(\zt^m)^2=1$, so
   $\zt\in\mu_n$.  On the other hand, if $\zt^k=1$ then
   $(-\zt)^{2k}=(\zt^k)^2=1$, so $2k$ is divisible by $m$.  As $m$ is
   odd this can only happen if $k$ is divisible by $m$, say $k=mj$.
   This means that $\zt^k=(\zt^m)^j=(-1)^j$, but we also assumed that
   $\zt^k=1$, so $j$ must be even.  As $k=mj$ this means that $k$ is
   divisible by $2m=n$.  This shows that $\zt\in\mu_n^\tm$.

   Next, $\vph_m(x)$ is the product of the terms $x-\zt$ for
   $\zt\in\mu_m^\tm$, so $\vph_m(-x)$ is the product of the
   corresponding terms $-x-\zt$.  The number of terms here is
   $|\mu_m^\tm|$, which is even, by part~(a).  It therefore does not
   matter if we change all the signs, so $\vph_m(x)$ is the product of
   the terms $x+\zt$.  Now $x+\zt=x-(-\zt)$, and
   $\{-\zt\st\zt\in\mu_m^\tm\}=\mu_n^\tm$, so we see that
   $\vph_m(-x)=\vph_n(x)$.

  \item[(c)] We can write $n=p^2m$ for some $m$, so $n/p=mp$.
   Suppose that $\zt\in\mu_n^\tm$.   Then
   $(\zt^p)^{mp}=\zt^n=1$.  On the other hand, if
   $(\zt^p)^k=\zt^{pk}=1$, then $pk$ must be divisible by $p^2m$, so
   $k$ must be divisible by $pm$.  It follows that
   $\zt^p\in\mu_{pm}^\tm$.

   Conversely, suppose that $\zt^p\in\mu_{mp}^\tm$.  It is then clear
   that $\zt^n=(\zt^p)^{mp}=1$, so $\zt\in\mu_n$.  On the other hand,
   suppose that $\zt^k=1$.  Then $(\zt^p)^k=1$, so $k$ is divisible by
   $mp$, say $k=mpj$.  Now the original relation $\zt^k=1$ can be
   written as $(\zt^p)^{mj}=1$, so $mj$ must be divisible by $mp$, say
   $mj=mpi$.  It follows that $k=mpj=p.mj=mp^2i=ni$, so $k$ is
   divisible by $n$.  This shows that $\zt\in\mu_n^\tm$ as claimed.

   Now note that $\vph_{n/p}(x^p)$ is the product of the terms
   $x^p-\xi$ for $\xi\in\mu^\tm_{n/p}$.  Here $x^p-\xi$ can be
   rewritten as the product of the terms $x-\zt$, as $\zt$ runs over
   the $p$th roots of $\xi$.  Thus, $\vph_{n/p}(x^p)$ is the product
   of all terms $x-\zt$ for which $\zt^p\in\mu_{n/p}^\tm$, or
   equivalently (by what we just proved) $\zt\in\mu_n^\tm$.  This
   means that $\vph_{n/p}(x^p)=\vph_n(x)$.

  \item[(d)] If we start with $\vph_p(x)$ and apply~(c) repeatedly we
   can find $\vph_{p^k}(x)$ for all $k$ (and any prime $p$).  If $p$
   is odd we can then use~(b) to find $\vph_{2p^k}(x)$, and then we
   can use method~(c) at the prime $2$ to find $\vph_{4p^k}(x)$,
   $\vph_{8p^k}(x)$ and so on.  Eventually this gives
   $\vph_{2^ip^j}(x)$ for all $i$ and $j$.  If $p$ and $q$ are
   distinct odd primes, then we cannot find $\vph_{pq}(x)$ by this
   method.  In particular, the first case that we do not cover is
   $\vph_{15}(x)$.  However, if we compute $\vph_{pq}(x)$ by some
   other method then using~(b) and~(c) we can find
   $\vph_{2^ip^jq^k}(x)$.

  \item[(e)] Let $N$ be the smallest number such that $\vph_N(x)$ has
   a coefficient not in $\{0,1,-1\}$.  If $N$ is divisible by $p^2$
   for some prime $p$, then $\vph_N(x)=\vph_{N/p}(x^p)$ by~(c).  Here
   $N/p<N$ so (by the definition of $N$) the coefficients of
   $\vph_{N/p}(x)$ are all in $\{0,1,-1\}$.  It follows that the same
   is true of $\vph_{N/p}(x^p)$, which gives a contradiction.  Thus,
   $N$ cannot be divisible by $p^2$ for any $p$, so $N$ is a product
   of distinct primes.  If one of these primes is $2$ then the
   remaining primes are odd, so~(b) is applicable and
   $\vph_N(x)=\vph_{N/2}(-x)$, which again gives a contradiction.
   Thus, $N$ must be a product of distinct odd primes.  There must be
   more than one prime factor, because of the rule
   $\vph_p(x)=\sum_{i=0}^{p-1}x^i$.

  \item[(f)] The first few numbers that are products of at least two
   odd primes are
   \[ 15, 21, 33, 35, 39, 51, 65, 69, 77, 85, 87, 91, 93, 95, 105.
   \]
   We can ask Maple to calculate the corresponding cyclotomic
   polynomials, and we find that they all have coefficients in
   $\{0,1,-1\}$ until we get to $\vph_{105}(x)$.  This has degree $48$
   and involves $-2t^7$ and $-2t^{41}$, so $N=105$.  In fact
   $105=3\tm 5\tm 7$, which is the smallest number that is a product
   of three distinct odd primes.

   Alternatively, we can make Maple do all the work automatically, as
   follows:
\begin{verbatim}
 for n from 1 to 1000 do
  f := numtheory[cyclotomic](n,x);
  A := {coeffs(f,x)} minus {0,1,-1};
  if nops(A) > 0 then
   print([n,sort(f)]);
   break;
  fi:
 od:
\end{verbatim}
 \end{itemize}
\EndDeferredSolution

\BeginDeferredSolution{ex-phi-pq}{8.9}
 We can reorganise the definition and use the geometric progression
 formula as follows:
 \begin{align*}
   f(x) &= (1-x)\left(\sum_{i=0}^{q-1}x^{ip}\right)
                \left(\sum_{j=0}^{p-1}x^{jq}\right)
                \left(\sum_{k=0}^\infty x^{kpq}\right) \\
   &= (1-x)\frac{x^{pq}-1}{x^p-1}
       \frac{x^{pq}-1}{x^q-1}\frac{1}{1-x^{pq}}
    = \frac{(x-1)(x^{pq}-1)}{(x^p-1)(x^q-1)} \\
   &= \frac{\vph_1(x)\vph_{pq}(x)\vph_p(x)\vph_q(x)\vph_1(x)}
           {\vph_p(x)\vph_1(x)\vph_q(x)\vph_1(x)}
    = \vph_{pq}(x).
 \end{align*}

 Now consider an arbitrary natural number $m$.  The element
 $m/p\in\F_q$ is represented by some $i\in\{0,\dotsc,q-1\}$, and the
 element $m/q\in\F_p$ is represented by some $j\in\{0,\dotsc,p-1\}$.
 We find that $m-(ip+jq)$ is divisible by both $p$ and $q$, so
 $m=ip+jq+kpq$ for some $k\in\Z$.  We define $\lm(m)$ to be $1$ if
 $k\geq 0$, and $0$ if $k<0$.  Note that $ip+jq\leq(q-1)p+(p-1)q<2pq$,
 so $\lm(m)=1$ for $m\geq 2pq$.  The definition of $f(x)$ can now be
 rewritten as
 \[ f(x) = \sum_{m=0}^\infty \lm(m)(x^m-x^{m+1}) =
     \sum_{m=0}^\infty (\lm(m)-\lm(m-1)) x^m.
 \]
 It follows that all the coefficients of $f(x)$ are in $\{0,1,-1\}$.
 We also see that for $m>2pq$ we have $\lm(m)-\lm(m-1)=1-1=0$, so
 $f(x)$ is a polynomial as expected.
\EndDeferredSolution

\BeginDeferredSolution{ex-fifth-root}{8.10}
 \begin{itemize}
  \item Any automorphism is uniquely determined by its effect on $\al$
  and on $\zt$. The image of $\al$ must be a root of $x^5-2$, so
  must be one of $\al$, $\zt\al$, $\zt^2\al$, $\zt^3\al$
  or $\zt^4\al$. In the same way, the image of $\zt$ must be another
  primitive 5th root of unity, i.e., a root of $\vph_5$, so is one of
  $\zt$, $\zt^2$, $\zt^3$ or $\zt^4$. This gives 20 possible
  automorphisms, $\theta_{ij}$ say, defined by
  \begin{eqnarray*}
  \theta_{ij}(\zt)&=&\zt^i\\
  \theta_{ij}(\al)&=&\zt^j\al
  \end{eqnarray*}
  for $i=1$, 2, 3 or 4 and $j=0$, 1, 2, 3 or 4.
  As the extension
  $\Q(\zt,\al)/\Q$ is Galois and has degree 20, these are all of
  the automorphisms.
  \item The automorphism $\psi$ which fixes $\zt$ and maps
  $\al$ to $\zt\al$ is clearly of order 5. The automorphism
  $\phi$ which fixes $\al$ and maps $\zt$ to $\zt^2$ is of order 4
  because $\phi^2(\zt)=\phi(\zt^2)=\zt^4$, and so
  $\phi^4(\zt)=\phi^2(\zt^4)=(\zt^4)^4=\zt$.

  The group generated by
  $\phi$ and $\psi$ has as subgroups $\langle\phi\rangle$ and $\langle\psi\rangle$
  so its order must be a multiple of 4 and of 5 by Lagrange's Theorem. It follows
  that this group must have order 20, so is the whole Galois group.
  \item We have:
  \begin{eqnarray*}
  &\phi\psi\phi^{-1}(\al)=\phi\psi(\al)=\phi(\zt\al)=\phi(\zt)\phi(\al)=\zt^2.\al\\
  &\phi\psi\phi^{-1}(\zt)=\phi\psi(\zt^3)=\phi(\zt^3)=\zt
  \end{eqnarray*}
  It follows that $\phi\psi\phi^{-1}=\psi^2$.
  \item We see that
  $$\zt^2+\zt+1+\zt^{-1}+\zt^{-2}=0.$$
  Rearranging, we get
  $$(\zt+\frac{1}{\zt})^2+(\zt+\frac{1}{\zt})-1=0.$$
  It follows that $\bt$ is a root of $X^2+X-1$, and so
  $\bt=\frac{-1\pm\sqrt{5}}{2}$, from the quadratic formula. It is then
  easy to see that $\Q(\bt)=\Q(\sqrt{5})$.

  $[\Q(\bt):\Q]=2$, so the index of the corresponding subgroup of
  $\Gal(M/\Q)$ must be 2, so its order must be 10.
  \item The group $\langle\phi^2,\psi\rangle$ is of order 10 (it contains
  an element of order 2, and an element of order 5, so its order must be a
  multiple of 10~--~but it isn't the whole group, as it doesn't contain $\phi$).
  Let $G$ be the subgroup associated to $\Q(\bt)$.
  If we can show that $\bt$ is fixed by both $\phi^2$ and by $\psi$, we
  will know that $\langle\phi^2,\psi\rangle\subseteq G$. But by the
  previous part of the question, $|G|=10$, and so we have to have
  $G=\langle\phi^2,\psi\rangle$, as required.

  But this is easy to check:
  \begin{eqnarray*}
  &\phi^2(\bt)=\phi^2(\zt)+\frac{1}{\phi^2(\zt)}=\zt^{-1}+\frac{1}{\zt^{-1}}=\frac{1}{\zt}+\zt=\bt\\
  &\psi(\bt)=\psi(\zt)+\frac{1}{\psi(\zt)}=\zt+\frac{1}{\zt}=\bt.
  \end{eqnarray*}
 \end{itemize}
\EndDeferredSolution

\BeginDeferredSolution{ex-forty-two}{8.11}
 \begin{itemize}
  \item $L=\Q(\al,\zt)$, where $\zt=e^{{2\pi i}/{7}}$ and
  $\al$ is the real 7th root of 3. Any automorphism must send $\zt$ to
  another primitive 7th root of unity, and send $\al$ to a 7th root
  of 3.

  There is an automorphism
  $\psi$ which fixes $\zt$ but maps $\al$ to $\zt\al$. Clearly
  $\psi$ is of order 7, as doing $\psi$ seven times fixes $\al$.

  Further, there is an automorphism $\phi$ which fixes $\al$ but sends
  $\zt$ to $\zt^3$. Applying $\phi$ successively to $\zt$ we see that
  $\zt$ is sent successively to
  $$\zt\mapsto\zt^3\mapsto\zt^2\mapsto\zt^6\mapsto\zt^4\mapsto\zt^5\mapsto\zt\mapsto\cdots$$
  so $\phi$ has order 6.
  \item
  Further,
  $$\phi\psi\phi^{-1}(\al)=\phi\psi(\al)=\phi(\zt\al)=\phi(\zt)\phi(\al)=\zt^3\al=\psi^3(\al)$$
  and
  $$\phi\psi\phi^{-1}(\zt)=\phi\psi(\zt^5)=\phi(\zt^5)=\zt=\psi^3(\zt)$$
  Thus $\phi\psi\phi^{-1}=\psi^3$.
  \item
  Finally, it remains to see that $\phi$ and $\psi$ generate the whole Galois
  group. But the Galois group has order 42, and the subgroup generated by
  $\phi$ and $\psi$ has order which is a multiple of both 6 and 7, so it must
  be the whole group.
 \end{itemize}
\EndDeferredSolution

\BeginDeferredSolution{ex-cyclic-five}{9.1}
 By the general theory of finite fields, we see that $\F_{11}^\tm$ is
 cyclic of order $10$, generated by some element $\al$ say.  It
 follows that the subgroup generated by $\al^2$ is cyclic of order
 $5$.

 In general, if $K$ is a finite field then $|K^\tm|+1=|K|$, which is a
 power of a prime.  As $5+1$ is not a power of a prime, we see that
 $|K^\tm|$ cannot be $5$, so $K^\tm$ cannot be isomorphic to $C_5$.
\EndDeferredSolution

\BeginDeferredSolution{ex-F-nine}{9.2}
 In $\F_3$ we have $\vph_8(0)=1\neq 0$ and $\vph_8(\pm 1)=2=-1\neq 0$,
 so $\vph_8(t)$ has no roots in $\F_3$, and thus has no factors of
 degree one in $\F_3[t]$.  Thus, the only way it can factor is as the
 product of two quadratic polynomials, say
 \[ t^4+1 = (t^2+at+b)(t^2+ct+d) =
     t^4+(a+c)t^3+(b+d+ac)t^2+(ad+bc)t+bd.
 \]
 By comparing coefficients we get
 \begin{align*}
  a+c &= 0 \\
  b+d+ac &= 0 \\
  ad+bc &= 0 \\
  bd &= 1.
 \end{align*}
 The last equation shows that $b\neq 0$, so $b=\pm 1$, so $b^2=1$.  We
 can thus multiply the last equation by $b$ to see that $d=b$.  On the
 other hand, the first equation gives $c=-a$.  Substituting these into
 the second equation and rearranging gives $b=-a^2$.  Here
 $a\in\{0,1,-1\}$ so $-a^2\in\{0,-1\}$ but we already know that
 $d=b\neq 0$ so $d=b=-1$.  As $b=-a^2$ we have $a\in\{1,-1\}$, and we
 have seen that $c=-a$.  We can arbitrarily choose to take $a=1$ and
 then $c=-1$, so we have the factorisation
 \[ \vph_8(t)=t^4+1=(t^2+t-1)(t^2-t-1) \in F_3[t]. \]
 This gives two fields of order 9:
 \begin{align*}
  K &= \F_3[\al]/(\al^2+\al-1) \\
  L &= \F_3[\bt]/(\bt^2-\bt-1).
 \end{align*}
 Now consider the field $\F_3[i]$ and the group
 \[ \F_3[i]^\tm = \{ 1,-1,i,-i,1+i,1-i,-1+i,-1-i\} \simeq C_8. \]
 The elements $1,-1,i$ and $-i$ are the roots of $t^4-1$, so the
 remaining elements are roots of $(t^8-1)/(t^4-1)=t^4+1=\vph_4(t)$.
 One checks that the elements $1\pm i$ are roots of $t^2+t-1$, and the
 elements $-1\pm i$ are roots of $t^2-t-1$.  There is thus a unique
 isomorphism $\phi\:K\to\F_3[i]$ with $\phi(\al)=1+i$, and a unique
 isomorphism $\psi\:L\to\F_3[i]$ with $\psi(\bt)=-1-i=-\phi(\al)$.  It
 follows that the composite isomorphism $\psi^{-1}\phi\:K\to L$ sends
 $\al$ to $-\bt$.
\EndDeferredSolution

\BeginDeferredSolution{ex-F-twentyfive}{9.3}
 Put $\al=\bsm 1&1\\2&1\esm$, and identify each element $a\in\F_5$
 with the matrix $aI=\bsm a&0\\ 0&a\esm$.  The set $K$ then consists
 of all matrices $a+b\al$ with $a,b\in\F_5$.  It is clear that this is
 a vector space of dimension two over $\F_5$, and so has order
 $5^2=25$.  Next, observe that
 \begin{align*}
  \al^2 &= \bsm 1&1\\ 2&1\esm \bsm 1&1\\ 2&1\esm
         = \bsm 3&2 \\ 4&3 \esm \\
  2\al+1 &= 2\bsm 1&1\\ 2&1\esm + \bsm 1&0\\ 0&1\esm
          = \bsm 3&2 \\ 4&3 \esm = \al^2.
 \end{align*}
 It follows that
 \begin{align*}
  (a+b\al)(c+d\al) &=
    ac+(ad+bc)\al+bd\al^2 = ac+(ad+bc)\al+bd(2\al+1) \\
    &= (ac+bd)+(ad+bc+2bd)\al \in K,
 \end{align*}
 so $K$ is closed under multiplication.  We also see from the above
 formulae that $(a+b\al)(c+d\al)=(c+d\al)(a+b\al)$, so multiplication
 in $K$ is commutative.  The remaining parts of
 Definition~\ref{defn-field}(b) are standard properties of matrix
 addition and multiplication.  We therefore see that $K$ is a
 commutative ring.  All that is left is to check that it is a field.
 To see this, put $f(x)=x^2-2x-1\in\F_5[x]$, so $f(\al)=0$, so there
 is a unique homomorphism $\phi$ from the ring $K'=K[x]/f(x)$ to $K$
 with $\phi(x+K[x]f(x))=\al$.  We also have
 \begin{align*}
  f(0) &= -1 \\
  f(1) &= -2 \\
  f(2) &= -1 \\
  f(3) &= 2 \\
  f(4) &= 2
 \end{align*}
 so $f(x)$ has no roots in $\F_5$.  As it is quadratic and has no
 roots, it must be irreducible, so $K'$ is a field.  As $1,x$ gives a
 basis for $K'$ over $\F_5$, and $1,\al$ gives a basis for $K$ over
 $\F_5$, we see that $\phi$ is an isomorphism.  This means that $K$ is
 also a field.
\EndDeferredSolution

\BeginDeferredSolution{ex-cyclic-galois}{9.4}
 Proposition~\ref{prop-cyclotomic-galois} tells us that
 $G(\Q(\mu_p)/\Q)$ is isomorphic to $(\Z/p\Z)^\tm=\F_p^\tm$, which is
 cyclic of order $p-1$ by Corollary~\ref{cor-units-cyclic}.
\EndDeferredSolution

\BeginDeferredSolution{ex-seven-cubed}{9.5}
 We have $\F_7^\tm=\{-3,-2,-1,1,2,3\}$, and we check that
 \[ 3^0 =  1 \qquad 3^1 =  3 \qquad 3^2=2 \qquad
    3^3 = -1 \qquad 3^4 = -3 \qquad 3^5 = -2.
 \]
 It follows that $\F_7^\tm$ is a cyclic group of order $6$, generated
 by $3$.  It follows that for every $a\in\F_7^\tm$ we have $a^6=1$, so
 $(a^3)^2=1$.  Thus, if $b^2\neq 1$ then $b$ is not the cube of any
 element in $\F_7^\tm$.  In particular, $3$ is not a cube.  (We could
 also have checked this by just writing out the cubes of all
 elements.)  Thus, the polynomial $f(t)=t^3-3$ has not roots in
 $\F_7$.  Any nontrivial factorisation would have to involve a
 quadratic term and a linear term, which would thus give a root; so
 $f(t)$ must be irreducible.  We therefore have a field
 $K=\F_7[\al]/(\al^3-3)$ of order $7^3=343$.  Now $\al^3=3$ and
 $3^6=1$, so $\al^{18}=1$, but the whole group $K^\tm$ has order
 $342$, so $\al$ does not generate $K^\tm$.

 % Finally, recall (from the general theory of finite fields) that
 % $G(K/\F_7)=\{1,\sg,\sg^2\}$, where $\sg(x)=x^7$.  As $\al^3=3$ we
 % have $\sg(\al)=\al^7=(\al^3)^2\al=9\al=2\al$, and so
 % $\sg^2(\al)=\sg(2\al)=4\al$.  It follows that
 % $\al,\sg(\al),\sg^2(\al)$ is not a basis for $K$ over $\F_7$, so
 % $\al$ does not give rise to a normal basis.
\EndDeferredSolution

\BeginDeferredSolution{ex-factor-mod-five}{9.6}
 We first remark that $\F_5=\{-2,-1,0,1,2\}$, with $(\pm 1)^2=1$ and
 $(\pm 2)^2=4=-1$.  It follows that $2$ is a generator of $\F_5^\tm$.
 We also see that $2^3=8=-2$, so we can write
 \[ f(x) = (x^2)^3 + 2^3 = (x^2+2)(x^4-2x^2+4)
     = (x^2+2)(x^4-2x^2-1).
 \]
 We can thus take $g_1(x)=x^2+2$.  For the other two factors, suppose
 that $g_2(x)=x^2+ax+b$ and $g_3(x)=x^2+cx+d$.  We should then have
 \[ x^4-2x^2-1 = g_2(x)g_3(x) =
     x^4 + (a+c)x^3 + (b+d+ac) x^2 + (ad+bc)x +bd.
 \]
 By comparing coefficients, we get
 \begin{align*}
  a+c &= 0 \\
  b+d+ac &= -2 \\
  ad+bc &= 0 \\
  bd &= -1.
 \end{align*}
 If $a=0$ then these equations reduce to $c=0$ and $d=-2-b$ and
 $bd=-1$.  By checking through the five possible values of $b$, we see
 that these equations are inconsistent.  Thus, we must have
 $a\neq 0$.  The first equation gives $c=-a$, and we can feed this
 into the third equation to get $a(d-b)=0$, but $a\neq 0$ so $d=b$.
 The last equation now says that $b^2=-1$, and it follows that
 $b=\pm 2$.  The second equation can now be rearranged as $a^2=2b+2$.
 If $b=-2$ this gives $a^2=-2$, but $-2$ is not a square in $\F_5$, so
 this is impossible.  If $b=2$ then we get $a^2=6=1$, so $a=\pm 1$.
 We should therefore take
 \begin{align*}
  g_2(x) &= x^2 + x + 2 \\
  g_3(x) &= x^2 - x + 2.
 \end{align*}
 One can then check directly that $f(x)=g_1(x)g_2(x)g_3(x)$ as
 expected.

 Note that $2$ is not a square in $\F_5$, so it is certainly not a
 sixth power, so $f(x)$ has no roots in $\F_5$.  It follows that
 $g_i(x)$ has no roots, and a quadratic with no roots is irreducible,
 so the three factors $g_i(x)$ are irreducible as claimed.

 Now suppose we have an extension field $K$ and an element $\al\in K$
 with $g_i(\al)=0$.  Let $d$ be the multiplicative order of $\al$, so
 we have $\al^m=1$ if and only if $m$ is divisible by $d$.  As
 $g_i(x)$ is a factor of $f(x)$ we see that $f(\al)=0$, so $\al^6=2$,
 so $\al^{12}=4=-1$ and $\al^{24}=1$.  It follows that $d$ divides
 $24$ but $d$ does not divide $12$; the only possibilities are $d=8$
 or $d=24$.  In fact, if $g_1(\al)=0$ then $\al^2=-2$ and it follows
 easily that $\al^8=1$, so $d=8$.  On the other hand, if $g_2(\al)=0$
 or $g_3(\al)=0$ then $\al^8=\al^6\al^2=2\al^2=2(\pm\al-2)\neq 1$, so
 $d$ must be $24$.
\EndDeferredSolution

\BeginDeferredSolution{ex-Fpp}{9.7}
 As $f(\al)=0$ we have $\al^p=\al+1$.  We can raise this to the $p$th
 power (remembering that $(x+y)^p=x^p+y^p\pmod{p}$) to get
 $\al^{p^2}=\al^p+1$, and then use $\al^p=\al+1$ again to get
 $\al^{p^2}=\al+2$.  By continuing in the same way, we find that
 $\al^{p^k}=\al+k$ for all $k$.  In particular, for $0<k<p$ this gives
 $\al^{p^k}\neq\al$.

 Now let $g(x)$ be the minimal polynomial of $\al$ over $\F_p$, which
 is an irreducible factor or $f(x)$.  If $g(x)$ has degree $d$, we
 have $|K|=p^d$.  By the general theory of finite fields, we have
 $a^{p^d}=a$ for all $a\in K$.  In particular $\al^{p^d}=\al$, so by
 our first paragraph we must have $d\geq p$.  On the other hand,
 $g(x)$ divides $f(x)$ and $f(x)$ has degree $p$, so we must have
 $d\leq p$.  We deduce that $d=p$ and $f(x)=g(x)$, so $f(x)$ is
 irreducible.
\EndDeferredSolution

\BeginDeferredSolution{ex-closed-infinite}{9.8}
 Let $K$ be a finite field.  We then have $|K|=p^d$, for some prime
 $p$ and $d>0$.  We have seen that $a^{p^d}=a$ for all $a\in K$.  Put
 $f(x)=x^{p^d}-x+1\in K[x]$, so $f(a)=1$ for all $a\in K$.  It follows
 that $f(x)$ has no roots in $K$, so $K$ is not algebraically closed.
\EndDeferredSolution

\BeginDeferredSolution{ex-H-cap-K}{11.1}
 Put $A=G(L/(L^HL^K))\leq G$.  Every automorphism $\sg\in A$ acts as
 the identity on $L^HL^K$, so in particular it acts as the identity on
 $L^H\sse L$, which means that $A\leq G(L/L^H)=H$.  By the same
 argument we have $A\sse G(L/L^K)=K$, so in fact $A\sse H\cap K$.
 Conversely, suppose that $\sg\in H\cap K$.  Any element $a\in L^HL^K$
 can be written as $a=b_1c_1+\dotsb+b_rc_r$ with $b_i\in L^H$ and
 $c_i\in L^K$.  We have $\sg(b_i)=b_i$ (because $\sg\in H$) and
 $\sg(c_i)=c_i$ (because $\sg\in K$).  It follows that $\sg(a)=a$ for
 all $a\in L^HL^K$, so $\sg\in A$.  This means that $A=H\cap K$.  The
 Galois Correspondence tells us that for all $M$ with $K\leq M\leq L$
 we have $M=L^{G(L/M)}$.  By taking $M=L^HL^K$ we see that
 $L^HL^K=L^A=L^{H\cap K}$ as claimed.
\EndDeferredSolution

\BeginDeferredSolution{ex-vier}{11.2}
 Choose elements $\rho$ and $\sg$ that generate $G(L/K)$, so
 $G(L/K)=\{1,\rho,\sg,\rho\sg\}$ with $\rho^2=\sg^2=1$ and
 $\rho\sg=\sg\rho$.  Put $G=G(L/K)$ and
 \begin{align*}
  A &= \{1,\rho\} & B &= \{1,\sg\} & C &= \{1,\rho\sg\} \\
  M &= L^A & N &= L^B & P &= L^C.
 \end{align*}
 Then $A$, $B$ and $C$ are the only proper nontrivial subgroups of
 $G$, so $M$, $N$ and $P$ are the only fields strictly between $K$ and
 $L$.  As $G$ is abelian, we see that all subroups are normal, so $M$,
 $N$ and $P$ are normal over $\Q$, with Galois groups $G/A$, $G/B$ and
 $G/C$ respectively.  All of these are of order $2$.  As
 $\sg\not\in A$, we see that $\sg$ acts nontrivially on $M$, so we can
 choose $\mu\in M$ with $\sg(\mu)\neq\mu$.  It follows that the
 element $\al=\mu-\sg(\mu)$ is nonzero, and it satisfies
 $\sg(\al)=-\al$.  It follows that $\al\not\in K$, and
 $[M:K]=|G/A|=2$, so $1$ and $\al$ must give a basis for $M$ over $K$,
 so $M=K(\al)$.  We also have $\sg(\al^2)=\al^2$, and so $\al^2\in
 M^{G/A}=K$.  Similarly, there is an element $\bt\in N$ such that
 $1,\bt$ is a basis for $N$ over $K$, and $\rho(\bt)=-\bt$, and
 $\bt^2\in K$.  Note that $\rho(\al)=\al$ (as $\al\in M$) and
 $\sg(\bt)=\bt$ (as $\bt\in N$).  It follows that
 $\rho(\sg(\al\bt))=(-\al)(-\bt)=\al\bt$, so $\al\bt\in P$.

 We next claim that the list $1,\al,\bt,\al\bt$ is linearly
 independent over $K$.  To see this, suppose that
 $a=w+x\al+y\bt+z\al\bt$ for some $w,x,y,z\in K$.  We can use the
 above formulae to understand $\sg(a)$ and $\rho(a)$, and we find that
 \begin{align*}
  a+\rho(a)+\sg(a)+\rho\sg(a) &= 4w \\
  a+\rho(a)-\sg(a)-\rho\sg(a) &= 4x\al \\
  a-\rho(a)+\sg(a)-\rho\sg(a) &= 4y\bt \\
  a-\rho(a)-\sg(a)+\rho\sg(a) &= 4z\al\bt.
 \end{align*}
 Thus, if $w+x\al+y\bt+z\al\bt=0$ we see that $w=x=y=z=0$.  This shows
 that the list $\CB=1,\al,\bt,\al\bt$ is linearly independent list, but
 $\dim_K(L)=|G|=4$, so $\CB$ must actually be a basis.

 \begin{center}
  \begin{tikzpicture}[scale=2]
   \def\Ga{( 1.0, 0.0)}
   \def\Aa{( 0.0, 1.0)}
   \def\Bb{( 1.0, 1.0)}
   \def\Cc{( 2.0, 1.0)}
   \def\Ta{( 1.0, 2.0)}

   \begin{scope}
    \draw(0, 0.0) node{$4$};
    \draw(0, 1.0) node{$2$};
    \draw(0, 2.0) node{$1$};
   \end{scope}
   \begin{scope}[xshift=1cm]
    \draw \Ga node{$G$};
    \draw \Aa node{$A$};
    \draw \Bb node{$B$};
    \draw \Cc node{$C$};
    \draw \Ta node{$\{1\}$};
    \draw[<-,shorten <=11pt,shorten >=11pt] \Ga -- \Aa;
    \draw[<-,shorten <=11pt,shorten >=11pt] \Ga -- \Bb;
    \draw[<-,shorten <=11pt,shorten >=11pt] \Ga -- \Cc;
    \draw[<-,shorten <=11pt,shorten >=11pt] \Aa -- \Ta;
    \draw[<-,shorten <=11pt,shorten >=11pt] \Bb -- \Ta;
    \draw[<-,shorten <=11pt,shorten >=11pt] \Cc -- \Ta;
   \end{scope}
   \begin{scope}[xshift=4cm]
    \draw \Ga node{$\Q$};
    \draw \Aa node{$K(\al)$};
    \draw \Bb node{$K(\bt)$};
    \draw \Cc node{$K(\al\bt)$};
    \draw \Ta node{$L$};
    \draw[->,shorten <=11pt,shorten >=11pt] \Ga -- \Aa;
    \draw[->,shorten <=11pt,shorten >=11pt] \Ga -- \Bb;
    \draw[->,shorten <=11pt,shorten >=11pt] \Ga -- \Cc;
    \draw[->,shorten <=11pt,shorten >=11pt] \Aa -- \Ta;
    \draw[->,shorten <=11pt,shorten >=11pt] \Bb -- \Ta;
    \draw[->,shorten <=11pt,shorten >=11pt] \Cc -- \Ta;
   \end{scope}
  \end{tikzpicture}
 \end{center}
\EndDeferredSolution

\BeginDeferredSolution{ex-golden}{11.3}
 Since $\zt^4+\zt^3+\zt^2+\zt+1=0$, we have
 $\zt^2+\zt+1+\zt^{-1}+\zt^{-2}=0$. Since $\al^2=\zt^2+2+\zt^{-2}$, we
 see that $\al^2+\al-1=0$.  Thus $\al$ is one of the roots of
 $x^2+x-1=0$, namely, $\al=(-1\pm\sqrt{5})/2$.  However,
 $\zt+\zt^{-1}=\zt+\ov{\zt}=2\cos(2\pi/5)>0$, so we must have
 $\al=(-1+\sqrt{5})/2$.  It follows that
 $\sqrt{5}=2\al+1=2\zt+2\zt^{-1}+1$, so $\sqrt{5}=2\al+1\in\Q(\zt)$.

 Next, we have
 \[ \bt^2=\zt^2-2+\zt^{-2}=\al^2-4=
     \left(\frac{-1+\sqrt{5}}{2}\right)^2 - 4 =
     \frac{6-2\sqrt{5}}{4}-4 = - \frac{1+\sqrt{5}}{2}.
 \]
 We also observe that $\sin(2\pi/5)>0$, and recall that when $t<0$ the
 symbol $\sqrt{t}$ refers to the square root in the upper half plane;
 we thus have $\bt=\sqrt{-(1+\sqrt{5})/2}$.

 We now put $G=G(\Q(\mu_5)/\Q)$ and look at the subgroup lattice.
 We know that
 \[ G = G(\Q(\mu_5)/\Q)=\{\sg_k\st k\in (\Z/5\Z)^\tm\} =
     = \{\sg_{-2},\sg_{-1},\sg_{1},\sg_{2}\},
 \]
 and this is cyclic of order $4$, generated by $\sg_{2}$.  It follows
 that the only subgroups are the trivial group, the whole group, and
 the subgroup $A=\{\ov{1},\ov{-1}\}$.  This means that the only
 subfields are $\Q(\mu_5)$, $\Q$ and the intermediate field
 $M=\Q(\mu_5)^A$.  Now $\sg_{-1}$ exchanges $\zt$ and $\zt^{-1}$ so it
 fixes $\al$ and sends $\bt$ to $-\bt$.  We therefore see that
 $M=\Q(\al)=\Q(\sqrt{5})$, and that $\Q(\bt)$ cannot be $M$ so it must
 be all of $\Q(\zt)$.  (In fact, one can check that
 $\zt=(\bt-\bt^2-3)/2$, which shows more explicitly that
 $\Q(\bt)=\Q(\zt)$.)

 The lattices can now be displayed as follows:
 \begin{center}
  \begin{tikzpicture}[scale=2]
   \def\Ga{( 0.0, 0.0)}
   \def\Ha{( 0.0, 1.0)}
   \def\Ta{( 0.0, 2.0)}

   \begin{scope}
    \draw(0, 0.0) node{$4$};
    \draw(0, 1.0) node{$2$};
    \draw(0, 2.0) node{$1$};
   \end{scope}
   \begin{scope}[xshift=1cm]
    \draw \Ga node{$G$};
    \draw \Ha node{$H$};
    \draw \Ta node{$\{1\}$};
    \draw[<-,shorten <=11pt,shorten >=11pt] \Ga -- \Ha;
    \draw[<-,shorten <=11pt,shorten >=11pt] \Ha -- \Ta;
   \end{scope}
   \begin{scope}[xshift=4cm]
    \draw \Ga node{$\Q$};
    \draw \Ha node{$\Q(\sqrt{5})$};
    \draw \Ta node{$\Q(\sqrt{-(1+\sqrt{5})/2})$};
    \draw[->,shorten <=11pt,shorten >=11pt] \Ga -- \Ha;
    \draw[->,shorten <=11pt,shorten >=11pt] \Ha -- \Ta;
   \end{scope}
  \end{tikzpicture}
 \end{center}
\EndDeferredSolution

\BeginDeferredSolution{ex-mu-eleven}{11.4}
\ \\
 \begin{itemize}
  \item[(a)]
   Since $\zt^{10}=\zt^{-1}$ etc., we can rewrite the given equation as
   \[ \zt^5+\zt^4+\zt^3+\zt^2+\zt+1 +
      \zt^{-1}+\zt^{-2}+\zt^{-3}+\zt^{-4}+\zt^{-5} = 0.
   \]
   Now
   \[ \begin{array}{rrrrrrrrrrrrr}
       \bt   &= &&&&& \zt & & +\zt^{-1} &&&& \\
       \bt^2 &= &&&& \zt^2 & & + 2 & & +\zt^{-2} &&& \\
       \bt^3 &= &&& \zt^3 && +3\zt && +3\zt^{-1} && + \zt^{-3} && \\
       \bt^4 &= && \zt^4 && +4\zt^2 && +6 && +4\zt^{-2} && +\zt^{-4} &\\
       \bt^5 &= & \zt^5 && +5\zt^3 && +10\zt &&
                  +10\zt^{-1} && + 5\zt^{-3} && +\zt^{-5}.
      \end{array}
   \]
   By combining these, we find that $\bt^5+\bt^4-4\bt^3-3\bt^2+3\bt+1=0$.
  \item[(b)] We have
   \begin{align*}
    \gm^2 &= \zt^2+\zt^8+\zt^7+\zt^{10}+\zt^6+\\
     & \qquad 2(\zt^5+\zt^{10}+\zt^6+\zt^4+
                \zt^2+\zt^9+\zt^7+\zt^3+\zt+\zt^8)\\
     &= (-1-\zt-\zt^3-\zt^4-\zt^5-\zt^9)+2(-1)\\
     &= -3-\gm,
   \end{align*}
   so $\gm^2+\gm+3=0$.  Since $\gm$ is a root of $x^2+x+3=0$, we see
   that $\gm=(-1\pm\sqrt{-11})/2$.  The terms in $\gm$ are
   distributed in the complex plane as follows:
   \begin{center}
    \begin{tikzpicture}[scale=2]
     \draw[->] (-1.3,0) -- (1.3,0);
     \draw[->] (0,-1.3) -- (0,1.3);
     \fill[black!20] (  0:1) circle(0.03);
     \fill           ( 33:1) circle(0.03);
     \draw ( 33:1.2) node {$\zt$};
     \fill[black!20] ( 65:1) circle(0.03);
     \fill           ( 98:1) circle(0.03);
     \draw ( 98:1.2) node {$\zt^3$};
     \fill           (131:1) circle(0.03);
     \draw (131:1.2) node {$\zt^4$};
     \fill           (164:1) circle(0.03);
     \draw (164:1.2) node {$\zt^5$};
     \fill[black!20] (196:1) circle(0.03);
     \fill[black!20] (229:1) circle(0.03);
     \fill[black!20] (262:1) circle(0.03);
     \fill           (295:1) circle(0.03);
     \draw (295:1.2) node {$\zt^9$};
     \fill[black!20] (327:1) circle(0.03);
     \fill[black!20] (360:1) circle(0.03);
    \end{tikzpicture}
   \end{center}
   It is clear from this that the imaginary part of $\gm$ is positive,
   so $\gm=(-1+\sqrt{-11})/2$, so $\sqrt{-11}=2\gm+1$.  It is also
   clear from the definition that $\gm\in\Q(\zt)$, so
   $\sqrt{-11}\in\Q(\zt)$.
  \item[(c),(d)] The general cyclotomic theory says that
   $G(K/\Q)=\{\sg_k\st k\in(\Z/11)^\tm\}$.  We have
   \[ (\Z/11)^\tm =
       \{-5,-4,-3,-2,-1,1,2,3,4,5\}.
   \]
   The powers of $2$ mod $11$ are as follows:
   \[ 2^0=1,  \;\;
      2^1=2,  \;\;
      2^2=4,  \;\;
      2^3=-3, \;\;
      2^4=5,  \;\;
      2^5=-1, \;\;
      2^6=-2, \;\;
      2^7=-4, \;\;
      2^8=3,  \;\;
      2^9=-5, \;\;
      2^{10}=1.
   \]
   This shows that $(\Z/11)^\tm$ is cyclic of order $10$, generated
   by $2$, and thus $G(K/\Q)$ is cyclic of order $10$, generated by
   $\sg_2$.  We write
   \begin{align*}
    C_{10} &= G(K/\Q) = \ip{\sg_2} \\
    C_5 &= \ip{\sg_2^2} = \ip{\sg_4} =
      \{1,\sg_4,\sg_5,\sg_{-2},\sg_3\} \\
    C_2 &= \ip{\sg_2^5} = \ip{\sg_{-1}} = \{1,\sg_{-1}\} \\
    C_1 &= \{1\}.
   \end{align*}
   These are all the subgroups of the Galois group.  It follows that
   the only subfields of $K$ are $K^{C_{10}}=\Q$, $K^{C_5}$, $K^{C_2}$
   and $K^{C_1}=K$.  The terms in $\gm$ are precisely the orbit of
   $\zt$ under $C_5$, so $\gm\in K^{C_5}$, so $\sqrt{-11}\in K^{C_5}$.
   We also know that $[K^{C_5}:\Q]=|C_{10}|/|C_5|=2$, which is the
   same as the degree of $\Q(\sqrt{-11})$, so we must have
   $K^{C_5}=\Q(\sqrt{-11})$.  Similarly, we have
   \[ \sg_{-1}(\bt) =
      \sg_{-1}(\zt)+\sg_{-1}(\zt)^{-1}=\zt^{-1}+\zt=\bt,
   \]
   so $\bt\in K^{C_2}$, and it follows that $K^{C_2}=\Q(\bt)$.  The
   subgroup and subfield lattices can thus be displayed as follows:
   \begin{center}
    \begin{tikzpicture}[scale=2]
     \def\Ga{( 0.8, 0.0)}
     \def\Gb{( 0.0, 0.6)}
     \def\Gc{( 2.0, 1.6)}
     \def\Gd{( 1.2, 2.2)}

     \begin{scope}
      \draw(0, 0.0) node{$10$};
      \draw(0, 0.6) node{$5$};
      \draw(0, 1.6) node{$2$};
      \draw(0, 2.2) node{$1$};
     \end{scope}
     \begin{scope}[xshift=1cm]
      \draw \Ga node{$C_{10}$};
      \draw \Gb node{$C_5$};
      \draw \Gc node{$C_2$};
      \draw \Gd node{$\{1\}$};
      \draw[<-,shorten <=11pt,shorten >=11pt] \Ga -- \Gb;
      \draw[<-,shorten <=11pt,shorten >=11pt] \Ga -- \Gc;
      \draw[<-,shorten <=11pt,shorten >=11pt] \Gb -- \Gd;
      \draw[<-,shorten <=11pt,shorten >=11pt] \Gc -- \Gd;
     \end{scope}
     \begin{scope}[xshift=4cm]
      \draw \Ga node{$\Q$};
      \draw \Gb node{$\Q(\sqrt{-11})$};
      \draw \Gc node{$\Q(\bt)$};
      \draw \Gd node{$\Q(\mu_{11})$};
      \draw[->,shorten <=11pt,shorten >=11pt] \Ga -- \Gb;
      \draw[->,shorten <=11pt,shorten >=11pt] \Ga -- \Gc;
      \draw[->,shorten <=11pt,shorten >=11pt] \Gb -- \Gd;
      \draw[->,shorten <=11pt,shorten >=11pt] \Gc -- \Gd;
     \end{scope}
    \end{tikzpicture}
   \end{center}
 \end{itemize}
\EndDeferredSolution

\BeginDeferredSolution{ex-two-group}{11.5}
 Put $M_i=L^{H_i}$, so $L=M_0\supset M_1\supset\dotsb\supset M_r=K$.
 The Galois Correspondence tells us that $L$ is normal over $M_i$,
 with Galois group $H_i$ (so $[L:M_i]=2^i$) and $M_i$ is normal over
 $K$ (with Galois group $G/H_i$).  It follows that $[M_i:M_{i+1}]=2$,
 so the standard analysis of degree two extensions says that
 $M_i=M_{i+1}(\al_i)$ for some $\al_i$ with $\al_i^2\in M_{i+1}$.
 This means that $L=K(\al_0,\dotsc,\al_{r-1})$.  More precisely, for
 any subset $I\sse\{0,1,\dotsc,r-1\}$ we can let $\al_I$ denote the
 product of the elements $\al_i$ for $i\in I$.  We then find that
 these elements $\al_I$ give a basis for $L$ over $K$.

 This does not yet capture all the information that one might want, as
 revealed by the following question.  Suppose we have fields
 $K\subset K(\al_1)\subset K(\al_0,\al_1)$, with $\al_1^2\in K$ and
 $\al_0^2\in K(\al_1)$.  When is it true that $K(\al_0,\al_1)$ is
 normal over $K$?  This is usually false but sometimes true.  We do
 not know a good general criterion even in this case where $r=2$, let
 alone the case of general $r$.
\EndDeferredSolution

\BeginDeferredSolution{ex-classify-cubics}{12.1}
 We first claim that $g_0(x)$ is irreducible over $\Q$.  If not, it
 would have to have a monic linear factor, say $x-a$ with $a\in\Q$.
 Then Gauss's Lemma (Proposition~\ref{prop-gauss}) would tell us that
 $a\in\Z$.  We would also have $g_0(a)=0$, which rearranges to give
 $a(3-a^2)=1$, so $a$ divides $1$, so $a=\pm 1$.  However $g_0(1)$ and
 $g_0(-1)$ are nonzero, so this is impossible.  By essentially the
 same argument, $g_1(x)$ is irreducible over $\Q$.  This can also be
 proved by applying Eisenstein's criterion (with $p=3$) to $g_0(x-1)$
 and $g_1(x-1)$.

 We now see from the general theory that the Galois groups are either
 $A_3=C_3$ (if the discriminant is a square) or $\Sg_3$ (if the
 discriminant is not a square).  Using the formula in
 Remark~\ref{rem-disc-simple} we see that the discriminant of $g_0(x)$
 is $-4\tm(-27)-27=81=9^2$, whereas the discriminant of $g_1(x)$ is
 $-4\tm 27-27=-135$.  Thus, the Galois group for $g_0(x)$ is $A_3$, and
 the Galois group for $g_1(x)$ is $\Sg_3$.
\EndDeferredSolution

\BeginDeferredSolution{ex-cyclic-cubic}{12.2}
 The first claim can be checked using Maple as follows:
\begin{verbatim}
 r := 1 + q + q^2;
 f := (x) -> x^3 - (3*x - 2*q - 1)*r;
 g := (x) -> (x^3+3*q*x^2-3*(q+1)*x-(4*q^3+6*q^2+6*q+1));
 s := (x) -> x^2+q*x-2*r;
 expand(f(s(x)) - f(x)*g(x));
\end{verbatim}
 It is possible but painful to do this by hand; $f(s(x))$ has 25 terms
 when fully expanded.

 Now suppose we have $\al\in L$ with $f(\al)=0$, and we put
 $\bt=s(\al)\in\Q(\al)$.  We can substitute $x=\al$ in the relation
 $f(s(x))=f(x)g(x)$ to see that $f(\bt)=f(\al)g(\al)=0$, so $\bt$ is
 another root of $f(x)$.  Next, as $f(x)$ is assumed to be
 irreducible, it must be the minimal polynomial of $\al$, so
 $\Q(\al)\simeq\Q[x]/f(x)$.  This means that homomorphisms from
 $\Q(\al)$ to any field $M$ biject with roots of $f(x)$ in $M$.  In
 particular, we can take $M=\Q(\al)$ and we find that there is a
 homomorphism $\sg\:\Q(\al)\to\Q(\al)$ with $\sg(\al)=\bt$.

 We next claim that $\bt\neq\al$, or equivalently that $\al$ is not a
 root of the quadratic polynomial $s(x)-x$.  This is clear because the
 minimal polynomial of $\al$ is $f(x)$, which is cubic, so it cannot
 divide $s(x)-x$.  It follows that $f(x)$ is divisible in $\Q(\al)[x]$
 by $(x-\al)(x-\bt)$.  The remaining factor is a monic polynomial
 of degree $1$, so it must have the form $x-\gm$ for some
 $\gm\in\Q(\al)$.  We now have a splitting
 $f(x)=(x-\al)(x-\bt)(x-\gm)$, so $\Q(\al)$ is a splitting field for
 $f(x)$.  This means that it is normal, and the order of the Galois
 group is $[\Q(\al):\Q]=3$.  All groups of order $3$ are cyclic, and
 $\sg$ is a nontrivial element, so we must have
 $G(\Q(\al)/\Q)=\{1,\sg,\sg^2\}$.
\EndDeferredSolution

\BeginDeferredSolution{ex-inv-sq-sum}{12.3}
 First, we have
 \[ x^3+ux^2+vx+w = f(x) = (x-\al)(x-\bt)(x-\gm) =
     x^3 - (\al+\bt+\gm) x^2 + (\al\bt+\bt\gm+\gm\al) x - \al\bt\gm,
 \]
 so
 \begin{align*}
  u &= -\al-\bt-\gm \\
  v &= \al\bt+\bt\gm+\gm\al \\
  w &= -\al\bt\gm.
 \end{align*}
 It follows that
 \[ w^2p = \al^2\bt^2 + \bt^2\gm^2 + \gm^2\al^2. \]
 This is similar to $v^2$, but not equal to it.  More precisely, we
 have
 \[ v^2 = \al^2\bt^2 + \bt^2\gm^2 + \gm^2\al^2 +
          2(\al^2\bt\gm + \al\bt^2\gm + \al\bt\gm^2)
        = w^2p + 2uw.
 \]
 Rearranging this gives $p=v^2/w^2-2u/w$.
\EndDeferredSolution

\BeginDeferredSolution{ex-vandermonde}{12.4}
 \begin{itemize}
  \item[(a)] One approach is to simply expand everything out.
   Alternatively, we can recall the behaviour of determinants under
   row and column operations, and argue as follows:
   \[ \det\bsm 1&1&1\\ \al&\bt&\gm\\ \al^2&\bt^2&\gm^2\esm =
      \det\bsm 1&0&0\\ \al&\bt-\al&\gm-\al\\
                \al^2&\bt^2-\al^2&\gm^2-\al^2\esm =
      (\bt-\al)(\gm-\al)
      \det\bsm 1&0&0\\ \al&1 &1 \\
                \al^2&\bt+\al&\gm+\al\esm =
      (\bt-\al)(\gm-\al)(\gm-\bt) = \dl(f).
   \]
   (At the first stage we subtracted the first column from each of the
   other two columns, then we extracted factors of $\bt-\al$ and
   $\gm-\al$ from the second and third columns, then we calculated the
   final determinant directly.)
  \item[(b)] We have
   \[ \det(MM^T) = \det(M)\det(M^T) = \det(M)^2 = \dl(f)^2 = \Dl(f).
   \]
  \item[(c)] This is just a direct calculation:
   \[ \begin{pmatrix}
      1&1&1\\
      \al&\bt&\gm\\
      \al^2&\bt^2&\gm^2
     \end{pmatrix}
     \begin{pmatrix}
      1&\al&\al^2\\
      1&\bt&\bt^2\\
      1&\gm&\gm^2
     \end{pmatrix} =
     \begin{pmatrix}
      1+1+1&\al+\bt+\gm&\al^2+\bt^2+\gm^2\\
      \al+\bt+\gm&\al^2+\bt^2+\gm^2&\al^3+\bt^3+\gm^3\\
      \al^2+\bt^2+\gm^2&\al^3+\bt^3+\gm^3&\al^4+\bt^4+\gm^4.
     \end{pmatrix}
   \]
  \item[(d)] We have
   \[ S_2=\al^2+\bt^2+\gm^2 =
       (\al+\bt+\gm)^2-2(\al\bt+\bt\gm+\gm\al)=-2a,
   \]
   as $\al+\bt+\gm=S_1=0$ and $\al\bt+\bt\gm+\gm\al=a$.
  \item[(e)] Add the three equations to get
   \[ (\al^3+\bt^3+\gm^3)+a(\al+\bt+\gm)+b(1+1+1)=0, \]
   or $S_3+aS_1+bS_0=0$. Thus $S_3=-aS_1-bS_0$. Also, add
   \begin{align*}
    \al^4+a\al^2+b\al &= 0 \\
    \bt^4+a\bt^2+b\bt &= 0 \\
    \gm^4+a\gm^2+b\gm &= 0
   \end{align*}
   to get $S_4=-aS_2-bS_1$. Thus we conclude that
   \begin{align*}
     S_3 &= -3b\\
     S_4 &= 2a^2.
   \end{align*}
  \item[(f)] Substituting the values of $S_0,\ldots,S_4$ into the
   matrix in (c), we get:
   \[ MM^T =
      \begin{pmatrix}
       3   & 0   & -2a  \\
       0   & -2a & -3b  \\
       -2a & -3b & 2a^2
      \end{pmatrix}.
   \]
   By part~(b), $\Dl(f)$ is the determinant of this matrix, which can
   be evaluated directly to give $\Dl(f)=-(4a^3+27b^2)$.
 \end{itemize}
\EndDeferredSolution

\BeginDeferredSolution{ex-classify-quartics}{13.1}
 Using the formula in Proposition~\ref{prop-resolvent}, we see
 that the resolvent cubic for $f_0(x)$ is
 $x^3-32x-64=64((x/4)^3-2(x/4)-1)$.  In the notation of
 Exercise~\ref{ex-classify-cubics}, this is $64 g_0(x/4)$, so the
 Galois group is the same as for $g_0(x)$, namely $A_3$.  Using
 Remark~\ref{rem-irr-resolvent} we deduce that the Galois group for
 $f_0(x)$ is $A_4$.

 Similarly, the resolvent cubic for $f_1(x)$ is $64 g_1(x/4)$, and the
 Galois group for $g_1(x)$ is $\Sg_3$, so the Galois group for
 $f_1(x)$ is $\Sg_4$.
\EndDeferredSolution

\BeginDeferredSolution{ex-biquad-quartic}{13.2}
 The discriminant is
 \begin{align*}
   \prod_{i<j}(\al_i-\al_j)^2
    &= (\al_0-\al_1)^2(\al_0-\al_2)^2(\al_0-\al_3)^2
       (\al_1-\al_2)^2(\al_1-\al_3)^2(\al_2-\al_3)^2 \\
    &= (2\sqrt{5})^2 (2\sqrt{2})^2 (2\sqrt{2}+2\sqrt{5})^2
       (2\sqrt{2}-2\sqrt{5})^2 (2\sqrt{2})^2 (2\sqrt{5})^2 \\
    &= 2^{14}5^2 (\sqrt{5}+\sqrt{2})^2(\sqrt{5}-\sqrt{2})^2 \\
    &= 2^{14}5^2 (5-2)^2 = 2^{14} 3^2 5^2 = 3686400.
 \end{align*}
 The splitting field is $\Q(\sqrt{2},\sqrt{5})$, so the Galois group
 is $C_2\tm C_2$ by Proposition~\ref{prop-biquadratic}.
\EndDeferredSolution

\BeginDeferredSolution{ex-quartic-discriminant}{13.3}
 We merely sketch this.  The matrix $M$ is
 \[  \begin{pmatrix}
      1     & 1     & 1     & 1         \\
      \al   & \bt   & \gm   & \delta    \\
      \al^2 & \bt^2 & \gm^2 & \delta^2  \\
      \al^3 & \bt^3 & \gm^3 & \delta^3
     \end{pmatrix}.
 \]
 If we put $S_i=\al^i+\bt^i+\gm^i+\delta^i$, then
 \[ MM^T = \begin{pmatrix}
            S_0 & S_1 & S_2 & S_3\\
            S_1 & S_2 & S_3 & S_4\\
            S_2 & S_3 & S_4 & S_5\\
            S_3 & S_4 & S_5 & S_6
           \end{pmatrix}.
 \]
 From the factorisation $f(x)=(x-\al)(x-\bt)(x-\gm)(x-\dl)$ we obtain
 \begin{align*}
  \al+\bt+\gm+\dl &= 0 \\
  \al\bt+\al\gm+\al\dl+\bt\gm+\bt\dl+\gm\dl &= 0 \\
  \al\bt\gm+\al\bt\dl+\al\gm\dl+\bt\gm\dl &= -p \\
  \al\bt\gm\dl &= q.
 \end{align*}
 From this we deduce that $S_0=4$, $S_1=0$ and $S_2=0$.  To compute $S_3$, use
 \begin{align*}
  \al^3+\bt^3+\gm^3+\delta^3 &=
    S_1^3
    -3(\al^2\bt+\mbox{similar terms})
    -6(\al\bt\gm+\mbox{similar terms}) \\
  \al^2\bt+\mbox{similar terms} &=
    S_1(\al\bt+\mbox{similar terms})
    -3(\al\bt\gm+\mbox{similar terms}) \\
  \al\bt\gm+\mbox{similar terms} &= -p.
 \end{align*}
 Combining these, together with $S_1=0$, we see that $S_3=-3p$.
 Using the same trick as in Exercise~\ref{ex-vandermonde}, we get that
 \begin{align*}
  S_4 &= -(pS_1+qS_0) =-4q  \\
  S_5 &= -(pS_2+qS_1) =0    \\
  S_6 &= -(pS_3+qS_2) =-3p^2
 \end{align*}
 and so
 \[ \Dl(f) = \det
     \begin{pmatrix}
      4   & 0   & 0   & -3p    \\
      0   & 0   & -3p & -4q    \\
      0   & -3p & -4q & 0      \\
      -3p & -4q & 0   & -3p^2
     \end{pmatrix} = 27p^4+256q^3.
 \]
\EndDeferredSolution

\BeginDeferredSolution{ex-check-solvable}{15.1}
 The polynomials $f_0(x)$ and $f_2(x)$ are solvable by radicals, but
 $f_1(x)$, $f_3(x)$, $f_4(x)$ and $f_5(x)$ are not.  This can be
 proved as follows.
 \begin{itemize}
  \item $f_0(x)$ is $x$ times a quartic, and quartics are solvable by
   radicals.  (Maple says that the relevant Galois group is $\Sg_4$.)
  \item $f_1(x)$ is irreducible by Eisenstein's criterion at $p=5$.
   It also has precisely three real roots (approximately $-1.33,
   -0.51, 1.60$), as one can see by plotting or an argument with
   Rolle's Theorem and the Intermediate Value Theorem.  The Galois
   group is thus $\Sg_5$ by Corollary~\ref{cor-all-perms}, which means
   that $f_1(x)$ is not solvable by radicals.
  \item Put $g_2(x)=2x^3-10x+5$, so $f_2(x)=g_2(x^2)$.  As $g_2(x)$ is
   cubic, it is solvable by radicals.  If the roots of $g_2(x)$ are
   $\al$, $\bt$ and $\gm$, then the roots of $f_2(x)$ are
   $\pm\sqrt{\al}$, $\pm\sqrt{\bt}$ and $\pm\sqrt{\gm}$.  It follows
   that the splitting field for $f_2(x)$ is obtained from that for
   $g_2(x)$ by adjoining some square roots, which is a further radical
   extension; so $f_2(x)$ is solvable by radicals.  Maple says
   that the relevant Galois group is of order 48, isomorphic to the
   subgroup of $\Sg_6$ generated by $(1~2~3~4)$ and $(1~5)(3~6)$.
  \item We observe that $f_3(x)=x^5f_1(1/x)$, so the roots of $f_3(x)$
   are the inverses of the roots of $f_1(x)$.  This means that
   $f_3(x)$ has the same splitting field as $f_1(x)$, so the Galois
   group is again $\Sg_5$, so $f_3(x)$ is not solvable by radicals.
  \item $f_4(x)$ is irreducible by Eisenstein's criterion at $p=3$,
   and has precisely three real roots (close to $x=0$ and
   $x=\pm 4.5$).  We can again use Corollary~\ref{cor-all-perms} to
   see that the Galois group is $\Sg_5$ and the polynomial is not
   solvable by radicals.
  \item One can check that $f_5(x)=f_1(x)^2$, so $f_5(x)$ has the same
   roots and the same splitting field as $f_1(x)$, so it is not
   solvable by radicals.
 \end{itemize}
\EndDeferredSolution

\BeginDeferredSolution{ex-septic}{15.2}
 It will be enough to show that the Galois group of the splitting
 field is $\Sg_7$.  Using Corollary~\ref{cor-all-perms}, it will thus
 be enough to show that $f(x)$ is irreducible and has precisely five
 real roots.  Irreducibility follows from Eisenstein's criterion at
 $p=7$.  We can plot the graph using Maple, and we see that the roots
 are as required:
 \begin{center}
  \begin{tikzpicture}[xscale=1.5]
   \def\ff{-0.21+(1.05+(-0.42+(-0.7+0.3*\x)*\x)*\x)*\x*\x*\x*\x}
   \begin{scope}
    \draw[->] (-1.4,0) -- (2.4,0);
    \draw[->] (0,-3.3) -- (0,2.4);
    \draw (-1,0) -- (-1,-0.1);
    \draw ( 0,0) -- ( 0,-0.1);
    \draw ( 1,0) -- ( 1,-0.1);
    \draw ( 2,0) -- ( 2,-0.1);
    \draw (-1,-0.25) node{$-1$};
    \draw ( 1,-0.25) node{$ 1$};
    \draw[red,domain=-1.35:2.35,smooth,samples=200,variable=\x]
      plot ({\x},{\ff});
    \draw[dotted] (1,0) circle(0.15);
   \end{scope}
   \begin{scope}[xshift=3cm,xscale=3,yscale=10]
    \draw (0.85,0) -- (1.15,0);
    \draw[red,domain=0.85:1.15,smooth,samples=200,variable=\x]
      plot ({\x},{\ff});
    \draw[dotted] (1,0) circle(0.15);
   \end{scope}
  \end{tikzpicture}
 \end{center}
 More rigorously, we can check that
 \[ f'(x) = 210(x^6-2x^5-x^4+2x^3) = 210x^3(x-1)(x+1)(x-2), \]
 which has four real roots, at $-1,0,1,2$.  Rolle's Theorem says that
 between any two real roots of $f(x)$ there is a real root of
 $f'(x)$, so there are at most five real roots.  We also have
 \begin{align*}
   f(x)  &\to -\infty\qquad\mbox{as }x\to-\infty\\
   f(-1) &= 26    \\
   f(0)  &= -21   \\
   f(1)  &= 2     \\
   f(2)  &= -325  \\
   f(x)  &\to +\infty\qquad\mbox{as }x\to+\infty
 \end{align*}
  so (by the Intermediate Value Theorem) $f(x)$ has exactly five real
  roots.
\EndDeferredSolution

\BeginDeferredSolution{ex-affine-five}{15.3}
\ \\
 \begin{itemize}
  \item[(a)] First note that
   \[ \rho_{ab}(\rho_{cd}(u)) =
       a(cu+d)+b = (ac)u+(ad+b) = \rho_{ac,ad+b}(u).
   \]
   It follows that $U$ is closed under composition.  We also see that
   $\rho_{10}$ is the identity, and that $\rho_{1/a,-b/a}$ is an
   inverse for $\rho_{ab}$.  This means that $U$ is a subgroup of
   $\Sg_5$.  Now define $\pi\:U\to\F_5^\tm$ by $\pi(\rho_{ab})=a$.
   The above composition formula shows that
   $\pi(\rho_{ab}\rho_{cd})=ac=\pi(\rho_{ab})\pi(\rho_{cd})$, so $\pi$
   is a homomorphism.  For each $a\in\F_5^\tm$ we have an element
   $\rho_{a0}\in U$ with $\pi(\rho_{a0})=a$, so $\pi$ is surjective.
   The kernel is $V=\{\rho_{1b}\st b\in\F_5\}$, which is therefore a
   normal subgroup.  The First Isomorphism Theorem tells us that
   $U/V\simeq\F_5^\tm=\{-2,-1,1,2\}$, which is cyclic of order $4$,
   generated by $2$.  We also see from the composition formula that
   $\rho_{1b}\rho_{1d}=\rho_{1,b+d}$, so $\rho_{1b}=\rho_{11}^b$.  It
   follows that $V$ is cyclic of order $5$, generated by $\rho_{11}$.
  \item[(b)] Let $H$ be a subgroup of $\Sg_5$, and let $C$ be a normal
   subgroup of $H$ that is cyclic of order $5$.  Choose a generator
   $\sg$ for $C$.  This has order $5$, and by considering the possible
   cycle types in $\Sg_5$ we see that it must be a $5$-cycle, say
   $\sg=(p_0\;p_1\;p_2\;p_3\;p_4)$.  Let $\tht$ be the permutation
   that sends $i$ to $p_i$, and note that
   $\tht^{-1}\sg\tht=\rho_{11}$.  Put $H'=\tht^{-1}H\tht$ and
   $C'=\tht^{-1}C\tht$, so $C'$ is normal in $H'$.  As
   $\tht^{-1}\sg\tht=\rho_{11}$ we see that $C'=V$.  Now consider an
   arbitrary element $\tau\in H'$.  Put $b=\tau(0)\in\F_5$.  As $V$ is
   normal in $H'$ we see that $\tau\rho_{11}\tau^{-1}$ must be another
   generator for $V$, so $\tau\rho_{11}\tau^{-1}=\rho_{1a}$ for some
   $a\in\F_5^\tm$.  We now claim that $\tau=\rho_{ab}$, or
   equivalently that the permutation $\phi=\rho_{ab}^{-1}\tau$ is the
   identity.  Indeed, we have $\rho_{ab}(0)=b=\tau(0)$, so
   $\phi(0)=0$.  We also have
   \[ \rho_{ab}\rho_{11}\rho_{ab}^{-1}=\rho_{a,a+b}\rho_{1/a,-b/a}=
    \rho_{1a} = \tau\rho_{11}\tau^{-1},
   \]
   so $\phi\rho_{11}\phi^{-1}=\rho_{11}$.  This means that $\phi$
   commutes with $\rho_{11}$, and thus also with
   $\rho_{1m}=\rho_{11}^m$.  It follows that
   \[ \phi(m) = \phi(\rho_{1m}(0)) = \rho_{1m}(\phi(0)) =
       \rho_{1m}(0) = m,
   \]
   so $\phi$ is the identity as claimed, so $\tau=\rho_{ab}$.  As
   $\tau$ was an arbitrary element of $H'$, we conclude that
   $H'\sse U$, and so $H=\tht H'\tht^{-1}\sse\tht U\tht^{-1}$.
  \item[(c)] Now instead let $H$ be an arbitrary transitive subgroup
   of $\Sg_5$.  For any $x\in\F_5$, the orbit $Hx$ is then the whole
   set $\F_5$.  We have the standard orbit-stabiliser identity
   $|H|=|Hx|.|\stab_H(x)|=5|\stab_H(x)|$, so $|H|$ must be divisible
   by $5$.  Moreover, $|H|$ must divide $|\Sg_5|=120$, so it cannot be
   divisible by $5^2$.  Let $C$ be any Sylow $5$-subgroup of $H$; then
   $|C|=5$ is prime, so $C$ must be cyclic.  If $C$ is normal in $H$
   then $H$ is conjugate to a subgroup of $U$ by part~(b).  From now
   on we suppose that $C$ is not normal in $H$.  Sylow theory tells us
   that the Sylow subgroups of $H$ are precisely the conjugates of
   $C$, and that the number $n$ of such conjugates divides
   $|H|/|C|$ and is congruent to $1$ modulo $5$.  Moreover, as $C$ is
   not normal we have $n>1$, and $|H|/|C|$ must divide
   $|\Sg_5|/|C|=24$.  It follows that $n=6$, and this must divide
   $|H|/|C|$, so $|H|\in\{30,60,120\}$.  If $|H|=120$ then $H$ is all
   of $\Sg_5$.  If $|H|=60$ then $H$ has index two, so it is normal by
   a standard lemma.  It is not hard to deduce that $H=A_5$.

   \textbf{This just leaves the case where $|H|=30$. I think that
    there are no subgroups of order $30$ in $\Sg_5$, but this needs a
    proof. }
 \end{itemize}
\EndDeferredSolution

\BeginDeferredSolution{ex-special-sextic}{15.4}
 These are not too difficult to construct. Here is one way to do it:
 \begin{description}
  \item[1] Choose a cubic with two positive real roots and one negative real
  root. For example, $x^3-7x+6=(x+3)(x-1)(x-2)$.
  \item[2] Move this polynomial up or down the $y$-axis slightly to make it
  irreducible, but still ensuring that there are two positive and one
  negative real root. (If you do this cleverly, you will be able to use
  Eisenstein's criterion to check irreducibility!) For example,
  $x^3-7x+6-\frac{1}{6}=\frac{1}{6}(6x^3-42x+35)$ is irreducible by
  Eisenstein's criterion with $p=7$.
  \item[3] Now replace $x$ by $x^2$ to get a polynomial of degree 6. In our
  example, we can consider the polynomial $6x^6-42x^2+35$. Now this polynomial
  is still irreducible by Eisenstein with $p=7$, and its roots are the square
  roots of the roots of the cubic in step 2, two of which were positive,
  giving 4 real roots, and one negative, giving 2 imaginary roots. Finally, the
  Galois group cannot be $\Sg_6$, since the polynomial is solvable by radicals
  (the roots are just the square roots of the roots of the cubic, so are
  certainly expressible as radicals).
 \end{description}
\EndDeferredSolution
