\documentclass{amsart}
\usepackage{hyperref}
\usepackage{fullpage}
\usepackage{amsrefs}

\input xypic
\newdir{ >}{{}*!/-9pt/\dir{>}}

\renewcommand{\:}{\colon}

\newtheorem{theorem}{Theorem}[section]
\newtheorem{conjecture}[theorem]{Conjecture}
\newtheorem{lemma}[theorem]{Lemma}
\newtheorem{proposition}[theorem]{Proposition}
\newtheorem{corollary}[theorem]{Corollary}
\theoremstyle{definition}
\newtheorem{remark}[theorem]{Remark}
\newtheorem{definition}[theorem]{Definition}
\newtheorem{example}[theorem]{Example}
\newtheorem{construction}[theorem]{Construction}

\newtheorem{notation}{Notation}
\renewcommand{\thenotation}{} % make the notation environment unnumbered

%\numberwithin{equation}{subsection}

\begin{document}
\title{Resultants and discriminants}
\author{N.~P.~Strickland}

\maketitle 

Let $K$ be a commutative ring, and write $K[x]_{\leq n}$ for the set
of polynomials over $K$ of degree at most $n$.  We will consider $K^m$
to be the set of column vectors of length $m$ over $K$.  We let 
$\kp\:K[x]_{\leq n}\to K^{n+1}$ be the obvious isomorphism, given by 
\[ \kp(\sum_{i=0}^na_ix^i) = [a_0,\dotsc,a_n]^T. \]

Now consider polynomials $f(x)\in K[x]_{\leq n}$ and 
$g(x)\in K[x]_{\leq m}$.  We define an $(n+m)\tm(n+m)$ matrix
$S(f(x),g(x))$ (called the \emph{Sylvester matrix}) as follows: the
first $m$ columns are
$\kp(f(x)),\kp(x\,f(x)),\dotsc,\kp(x^{m-1}f(x))$, and the remaining
$n$ columns are $\kp(g(x)),\dotsc,\kp(x^{n-1}g(x))$.  For example, if
$n=2$ and $m=3$ and
\begin{align*}
 f(x) &= a+bx+cx^2 \\
 g(x) &= p+qx+rx^2+sx^3 
\end{align*}
then
\[ S(f(x),g(x)) =
   \left[\begin{array}{ccc|cc}
    a & 0 & 0 & p & 0 \\
    b & a & 0 & q & p \\
    c & b & a & r & q \\
    0 & c & b & s & r \\
    0 & 0 & c & 0 & s
   \end{array}\right].
\]
We then define $R(f(x),g(x))\in K$ to be the determinant of this
matrix.  This is called the \emph{resultant} of $f(x)$ and $g(x)$.  We
also define the \emph{discriminant} of $f(x)$ to be the element
\[ \Dl(f(x)) = (-1)^{(n^2-n)/2} R(f'(x),f(x)) \in K. \]

The main reason why these constructions are interesting is as follows:
\begin{theorem}\label{thm-resultant}
 Suppose that $f(x)=\prod_{i=1}^n(x-\al_i)$ and
 $g(x)=\prod_{j=1}^m(x-\bt_j)$.  Then 
 \begin{align*}
  R(f(x),g(x)) &=
   \prod_{i=1}^n\prod_{j=1}^m(\al_i-\bt_j) = 
   \prod_{i=1}^ng(\al_i) = (-1)^{nm}\prod_{j=1}^m f(\bt_j) \\
  \Dl(f(x)) &= (-1)^{(n^2-n)/2}\prod_{i\neq j}(\al_i-\al_j)
    = \prod_{i<j}(\al_i-\al_j)^2.
 \end{align*}
\end{theorem}

This will be proved after some preliminary results.



\begin{bibdiv}
\begin{biblist}
\bibselect{%
../../BiBTeX/refs,%
../../BiBTeX/myrefs%
}
\end{biblist}
\end{bibdiv}

\end{document}
