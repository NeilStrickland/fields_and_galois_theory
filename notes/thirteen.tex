\documentclass{amsart}
\usepackage{hyperref}
\usepackage{fullpage}

\newcommand{\F}         {{\mathbb{F}}}          
\newcommand{\Q}         {{\mathbb{Q}}}
\newcommand{\al}        {\alpha}
\newcommand{\ov}[1]     {\overline{#1}}
\newcommand{\sse}       {\subseteq}
\newcommand{\tm}        {\times}
\newcommand{\zt}        {\zeta}
\renewcommand{\:}{\colon}

\newtheorem{theorem}{Theorem}[section]
\newtheorem{conjecture}[theorem]{Conjecture}
\newtheorem{lemma}[theorem]{Lemma}
\newtheorem{proposition}[theorem]{Proposition}
\newtheorem{corollary}[theorem]{Corollary}
\theoremstyle{definition}
\newtheorem{remark}[theorem]{Remark}
\newtheorem{definition}[theorem]{Definition}
\newtheorem{example}[theorem]{Example}
\newtheorem{construction}[theorem]{Construction}

\newtheorem{notation}{Notation}
\renewcommand{\thenotation}{} % make the notation environment unnumbered

%\numberwithin{equation}{subsection}

\begin{document}
\title{The Galois theory of $\Q(\mu_{13})$}
\author{N.~P.~Strickland}

\maketitle 

Consider the field $K=\Q(\mu_{13})$.  This is generated by
$\zt=e^{2\pi i/13}$, which satisfies $\sum_{k=0}^{12}\zt^k=0$

The Galois group $G(K/\Q)$ is identified with $\F_{13}^\tm$ in the
usual way.  This is cyclic of order $12$, generated by $2$.  The
subgroups of the Galois group are as follows:
\begin{align*}
 C_1    &= \{1\} \\
 C_2    &= \{1,-1\} \\
 C_3    &= \{1,3,-4\} \\
 C_4    &= \{1,5,-1,-5\} \\
 C_6    &= \{1,3,4,-1,-3,-4\} \\
 C_{12} &= \{1,2,3,4,5,6,-1,-2,-3,-4,-5,-6\}.
\end{align*}
We write $K_d=K^{C_{12/d}}\sse K$, so $[K_d:\Q]=d$ and
$G(K_d/\Q)=C_{12}/C_{12/d}\simeq C_d$.  Put 
\[ \al_d = \sum_{k\in C_{12/d}} \zt^k \in K_d. \]
Explicitly, we have 
\begin{align*}
 \al_1 &= -1 \\
 \al_2 &= \zt + \zt^3 + \zt^4 + \zt^{-1} + \zt^{-3} + \zt^{-4} \\
 \al_3 &= \zt + \zt^{-1} + \zt^5 + \zt^{-5} \\
 \al_4 &= \zt + \zt^3 + \zt^{-4} \\
 \al_6 &= \zt + \zt^{-1} = 2\cos(2\pi/13) \\
 \al_{12} &= \zt.
\end{align*}

We also use the elements
\begin{align*}
 r &= \left(-\tfrac{13}{2}(5+3\sqrt{-3})\right)^{1/3} \\
 w &= \al_4 - \ov{\al_4} \\
 y &= 2\al_2 + 1 = \sum_{k=0}^{11} (-1)^k \zt^{2^k} \\
\end{align*}
In the case of $r$, we use the unique cube root that lies in the first
quadrant, so $r\simeq 0.91+3.49i$.  It can be shown that
\begin{align*}
 x &= (r+\ov{r})/\sqrt{13} \\
 y &= |r| = \sqrt{13} \\
 w &= \sqrt{(3\sqrt{13}-13)/2} \\
 \al_2 &= (\sqrt{13}-1)/2 \\
 \al_3 &= (r+\ov{r}-1)/3 \\
 \al_4 &= \tfrac{1}{4}(\sqrt{13}-1) +
          \tfrac{1}{2}\sqrt{(3\sqrt{13}-13)/2} \\
 K_2 &= \Q(\sqrt{13}) \\
 K_3 &= \Q(r+\ov{r}) \\
 K_4 &= \Q(\sqrt{(3\sqrt{13}-13)/2}) \\
 K_6 &= \Q(x).
\end{align*}


\end{document}
