\documentclass{amsart}
\usepackage{hyperref}
\usepackage{fullpage}
\usepackage{amsrefs}

\usepackage[matrix,arrow]{xy}
\usepackage{xypdf}
\newdir{ >}{{}*!/-9pt/\dir{>}}

\newcommand{\al}        {\alpha}
\newcommand{\dl}        {\delta}
\newcommand{\tht}       {\theta}
\newcommand{\om}        {\omega}
\newcommand{\omb}	{\overline{\omega}}

\newcommand{\Dl}        {\Delta}

\renewcommand{\:}{\colon}

\newtheorem{theorem}{Theorem}[section]
\newtheorem{conjecture}[theorem]{Conjecture}
\newtheorem{lemma}[theorem]{Lemma}
\newtheorem{proposition}[theorem]{Proposition}
\newtheorem{corollary}[theorem]{Corollary}
\theoremstyle{definition}
\newtheorem{remark}[theorem]{Remark}
\newtheorem{definition}[theorem]{Definition}
\newtheorem{example}[theorem]{Example}
\newtheorem{construction}[theorem]{Construction}

\newtheorem{notation}{Notation}
\renewcommand{\thenotation}{} % make the notation environment unnumbered

%\numberwithin{equation}{subsection}

\begin{document}
\title{Cubics}
\author{N.~P.~Strickland}

\maketitle 

Consider a cubic $f(x)=x^3+bx+c$ with $b\neq 0$.  Put
$\Dl=-4b^3-27c^2$.  

\begin{proposition}\label{prop-discriminant}
 We have $\Dl=0$ iff $f(x)$ has a repeated root.  In particular, if
 $f(x)$ is irreducible then $\Dl\neq 0$.
\end{proposition}
\begin{proof}
 First, suppose that $\Dl=0$, so $27c^2=-4b^3$.  Put $\al=-(3c)/(2b)$,
 and note that 
 \begin{align*}
  \al^2 &= \frac{9c^2}{4b^2}= \frac{27c^2/3}{4b^2}
         = \frac{-4b^3/3}{4b^2} = -b/3 \\
  \al^3 &= \frac{-27c^3}{8b^3} = \frac{4b^3c}{8b^3} = c/2 \\  
  (x-\al)^2(x+2\al) &= (x^2-2\al x+\al^2)(x+2\al) \\
    &= x^3 - 3\al^2 x + 2\al^3 = x^3+bx+c = f(x),
 \end{align*}
 so $f(x)$ has a repeated root at $\al$.  

 For the converse, one can just expand everything out to check that 
 \[ (18bx-27c)f(x)+(9xc-4b^2-6bx^2)f'(x) = \Dl. \]
 In particular, the left hand side is actually a constant, independent
 of $x$.  If $f(x)$ has a repeated root at $\al$ then
 $f(\al)=f'(\al)=0$ so we can substitute $x=\al$ in the above to
 get $0=\Dl$. 
\end{proof}

From now on we assume that $f(x)$ has no repeated roots, so 
$\Dl\neq 0$.  We put $\dl=\sqrt{\Dl}$.  For definiteness, if 
$\Dl\geq 0$ we take $\dl$ to be the nonnegative square root of $\Dl$,
and if $\Dl<0$ then we take $\dl$ to be the square root of $\Dl$ lying
in the upper half-plane.  Next, we put
\begin{align*}
 m &= (-27c + \sqrt{-27}\dl)/2 \\
 n &= (-27c - \sqrt{-27}\dl)/2,
\end{align*}
so 
\begin{align*}
 m+n &= -27c \\
 m-n &= \sqrt{-27}\dl \\
 mn &= \tfrac{1}{4}((-27c)^2-(\sqrt{-27}\dl)^2) 
     = \tfrac{1}{4}(27^2c^2+27(-4b^3-27c^2)) \\
    &= -27b^3 = (-3b)^3.  
\end{align*}
As $mn=-b^3$ and $b\neq 0$ we have $m,n\neq 0$.

Now let $\mu_0$ be one of the cube roots of $m$.  For definiteness, we
can take $\mu_0$ be the unique cube root of $m$ that can be written as
$\mu_0=re^{i\tht}$ with $0\leq\tht<2\pi/3$.  As $m\neq 0$ we also have
$\mu_0\neq 0$.

Now put 
\[ \om = e^{2\pi i/3} = \cos(2\pi/3) + i\sin(2\pi/3) =
     \frac{\sqrt{3}i-1}{2},
\]
and note that $\om^3=1$ and 
\[ \om^2 = \omb = 1/\om = -1-\om = e^{-2\pi i/3} =
     \frac{-\sqrt{3}i-1}{2}.
\]
The other two cube roots of $m$ are then $\mu_1=\om\mu_0$ and
$\mu_2=\omb\mu_0$.  We put 
\begin{align*}
 \nu_0 &= -3b/\mu_0 \\
 \nu_1 &= -3b/\mu_1 = \omb\nu_0 \\
 \nu_2 &= -3b/\mu_0 = \om\nu_0.
\end{align*}
As $\mu_k^3=m$ and $(-3b)^3=mn$ and $b\neq 0$ we find that
$\nu_k^3=n$ and $\nu_k\neq 0$.  We now put 
$\al_k=(\mu_k+\nu_k)/3$.  The binomial expansion gives
\[
 \al_k^3 = \tfrac{1}{27}
    (\mu_k^3 + 3\mu_k(\mu_k\nu_k) + 3\nu_k(\mu_k\nu_k) + \nu_k^3).
\]
Here $\mu_k^3=m$ and $\nu_k^3=n$, so $\mu_k^3+\nu_k^3=m+n=-27c$.  On
the other hand, we have $\mu_k\nu_k=-3b$ by the definition of
$\nu_k$.  This means that
\begin{align*}
 \al_k^3 &= \tfrac{1}{27} (m - 9b\mu_k -9b\nu_k + n) \\
  &= \tfrac{1}{27}(-27c - 27b \frac{\mu_k+\nu_k}{3}) \\
  &= -c-b\al_k,
\end{align*}
so $f(\al_k)=\al_k^3+b\al_k+c=0$.  We thus have three roots of $f(x)$
which appear to be different, but we need to check that there is not
some hidden reason why they could be the same.  This will follow from
the formula below.

\begin{proposition}\label{prop-vandermonde}
 With $\dl$ and $\al_k$ defined as above, we have 
 \[ \dl = (\al_0-\al_1)(\al_0-\al_2)(\al_2-\al_1). \]
\end{proposition}
\begin{proof}
 First, from the definition of the numbers $\al_k$ and some easy
 manipulation we have 
 \begin{align*}
  3(\al_0-\al_1) &= \mu_0+\nu_0-\om\mu_0-\omb\nu_0 
                  = (1-\om)(\mu_0-\omb\nu_0) \tag{A} \\
  3(\al_0-\al_2) &= \mu_0+\nu_0-\omb\mu_0-\om\nu_0 
                  = (1-\omb)(\mu_0-\om\nu_0) \tag{B} \\
  3(\al_2-\al_1) &= \omb\mu_0+\om\nu_0-\om\mu_0-\om\nu_0 
                  = (\omb-\om)(\mu_0-\nu_0) \tag{D}
 \end{align*}
 We now want to multiply these together.  On the one hand, by fairly
 direct expansion we see that
 \[ (1-\om)(1-\omb)(\omb-\om) = -3\sqrt{3}i = -\sqrt{-27}. \tag{E} \]
 On the other hand, as the roots of $x^3-1$ are $1$, $\om$ and $\omb$
 we have 
 \[ (x-\omb)(x-\om)(x-1) = x^3-1. \]
 We saw earlier that $\nu_0\neq 0$, so it is legitimate to substitute
 $x=\mu_0/\nu_0$ and then multiply through by $\nu_0^3$ to get
 \[ (\mu_0-\omb\nu_0)(\mu_0-\om\nu_0)(\mu_0-\nu_0) = 
      \mu_0^3-\nu_0^3 = m-n = \sqrt{-27}\dl \tag{F}
 \]
 If we multiply~(A), (B) and (C), then use~(D) and~(E) to simplify the
 right hand side, we obtain 
 \[ 27 (\al_0-\al_1)(\al_0-\al_2)(\al_2-\al_1) = 
      -\sqrt{-27}\sqrt{-27}\dl = 27\dl.
 \]
 After dividing through by $27$, we get the claimed identity.
\end{proof}
   

\begin{bibdiv}
\begin{biblist}
\bibselect{%
../../BiBTeX/refs,%
../../BiBTeX/myrefs%
}
\end{biblist}
\end{bibdiv}

\end{document}
