\documentclass{amsart}
\usepackage{hyperref}
\usepackage{fullpage}
\usepackage{amsrefs}

\input xypic
\newdir{ >}{{}*!/-9pt/\dir{>}}

\newcommand{\C}         {{\mathbb{C}}}
\newcommand{\Q}         {{\mathbb{Q}}}
\newcommand{\R}         {{\mathbb{R}}}
\newcommand{\al}        {\alpha}
\newcommand{\bt}        {\beta} 
\newcommand{\gm}        {\gamma}
\newcommand{\half}      {{\textstyle\frac{1}{2}}}
\newcommand{\lm}        {\lambda}
\newcommand{\om}        {\omega}
\newcommand{\sg}        {\sigma}
\newcommand{\tht}       {\theta}

\renewcommand{\:}{\colon}

\newtheorem{theorem}{Theorem}[section]
\newtheorem{conjecture}[theorem]{Conjecture}
\newtheorem{lemma}[theorem]{Lemma}
\newtheorem{proposition}[theorem]{Proposition}
\newtheorem{corollary}[theorem]{Corollary}
\theoremstyle{definition}
\newtheorem{remark}[theorem]{Remark}
\newtheorem{definition}[theorem]{Definition}
\newtheorem{example}[theorem]{Example}
\newtheorem{construction}[theorem]{Construction}

\newtheorem{notation}{Notation}
\renewcommand{\thenotation}{} % make the notation environment unnumbered

%\numberwithin{equation}{subsection}

\begin{document}
\title{Extensions with Galois group $C_3$}
\author{N.~P.~Strickland}

\maketitle 

Fix a rational number $q$, and note that the number 
\[ r=q^2+q+1 = (q+1/2)^2+3/4 \] 
is always strictly positive.  Consider the polynomial
\[ f(x) = x^3 - (3x-2q-1)r. \]
Alternatively, this is the most general polynomial of the form
\[ f(x) = x^3 + ax^2 + bx + c \] 
with $b\neq 0$ but $a=9b^2+27c^2+4b^3=0$; the parameter $q$ can be
recovered as $q=-(b+3c)/(2b)$.

Note that the roots of $f'(x)$ are $\pm\sqrt{r}$, and that
\[ f(\pm\sqrt{r})=2(q+\half\pm\sqrt{r}) = 
     2\left((q+\half) \pm
              \sqrt{(q+\half)^2+\tfrac{3}{4}}\right). 
\] 
From this we check that $f(-\sqrt{r})>0>f(+\sqrt{r})$.  It follows
that there is a unique root $\al$ of $f(x)$ with $\al<-\sqrt{r}$, and
a unique root $\bt$ with $-\sqrt{r}<\bt<+\sqrt{r}$, and a unique root
$\gm$ with $+\sqrt{r}<\gm$.  We put $K=\Q(\al,\bt,\gm)\subset\R$,
which is a splitting field for $f(x)$ over $\Q$.
 
Another way to check that there are three distinct roots is to use the
identity 
\[ (2x+2q+1)(x\,f'(x)-3f(x)) - 4r\,f'(x) = 9r, \]
which is a nonzero constant.

Now put $s(x)=x^2+qx-2r$.  One can check that
$f(s(x))=f(x)g(x)$, where
\[ g(x) = x^3+3qx^2-3(q+1)x-(4q^3+6q^2+6q+1). \]
It follows that $s$ preserves the set $R=\{\al,\bt,\gm\}$ of roots of
$f(x)$.  One can also check that in $\Q[x]$ we have
\[ x+s(x)+s(s(x)) = (x+2q)f(x), \]
so $\tht+s(\tht)+s(s(\tht))=0$ whenever $\tht\in R$.  We can compare
this equation for $\tht$ with the corresponding equation for $s(\tht)$
to see that $s(s(s(\tht)))=\tht$.  It follows that $s$ acts on $R$
either as the identity or as a $3$-cycle.  The first option is
incompatible with the equation $\tht+s(\tht)+s(s(\tht))=0$ (because at
least two of the roots are nonzero).  It follows that $s$ acts as a
$3$-cycle, and thus that $K=\Q(\al)=\Q(\bt)=\Q(\gm)$.  In particular,
if any one of the roots is rational, then they all are.

From now on we suppose that the roots are all irrational, so $f(x)$ is
irreducible and $K\simeq\Q[x]/f(x)$.  In this context we see that
there is an automorphism $\sg$ of $K$ with $\sg(\tht)=s(\tht)$ for all
$\tht\in R$, and that $G(K/\Q)$ is cyclic of order $3$, generated by
$\sg$.  

Now put $\om=(\sqrt{-3}-1)/2\in\C$ and $L=\Q(\om)$ and $M=KL$.  The
usual theory of cyclic extensions tells us that the element
\[ \lm = \frac{\om-q}{3r}(\al + \om^2\sg(\al) + \om\sg^2(\al)) \]
satisfies $\lm^3\in L$ and $M=L(\lm)$.   In fact, one can check that
\[ \lm^3 = \frac{\om-q}{\om^2-q} = (\om-q)^2/r . \]


\begin{bibdiv}
\begin{biblist}
\bibselect{%
../../BiBTeX/refs,%
../../BiBTeX/myrefs%
}
\end{biblist}
\end{bibdiv}

\end{document}
